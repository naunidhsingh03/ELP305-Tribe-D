
\documentclass[table]{rapportCS}
\usepackage{lipsum}
\usepackage[french]{babel}
\usepackage{titlesec}
\usepackage{graphicx}
\usepackage{subcaption}
\usepackage{glossaries}
\usepackage[table]{xcolor}
\usepackage{multirow}
% \usepackage[utf8]{inputenc}
\usepackage{enumitem}
\usepackage[export]{adjustbox}
\usepackage{glossaries}
\usepackage[T1]{fontenc}
\usepackage{imakeidx}
\graphicspath{ {figures/} }
\usepackage[hypertexnames=false]{hyperref}
\makeindex
\indexsetup{othercode=\footnotesize}
\usepackage{listings}
\lstset{
  basicstyle=\small\ttfamily,
  language=C++,
  numbers=left,
  numberstyle=\tiny,
  numbersep=5pt,
  tabsize=2,
  extendedchars=true,
  breaklines=true,
  keywordstyle=\color{blue}\ttfamily,
  stringstyle=\color{red}\ttfamily,
  showspaces=false,
  showtabs=false,
  showstringspaces=false,
  backgroundcolor=\color{white},
  captionpos=b,
  frame=single,
  framexleftmargin=5mm,
  framexrightmargin=5mm,
  framextopmargin=2mm,
  framexbottommargin=2mm,
  xleftmargin=5mm,
  xrightmargin=5mm,
}

\usepackage{fancyhdr}
\pagestyle{fancy}
\fancyhf{} % Clear header and footer
\fancyhead[C]{}
\makeglossaries
\newglossaryentry{ABS}{
  name={ABS (Acrylonitrile Butadiene Styrene)},
  description={A strong and durable thermoplastic polymer used in manufacturing.}
}

\newglossaryentry{Acrylonitrile butadiene styrene}{
  name={Acrylonitrile butadiene styrene (ABS)},
  description={A common thermoplastic used to make light, rigid, molded products such as pipe, automotive body parts, wheel covers, enclosures, and protective headgear.}
}

\newglossaryentry{Adapter Voltage Rating}{
  name={Adapter Voltage Rating},
  description={This specifies the voltage range for the adapter that provides power to the soap water sprinkler.
}
}

\newglossaryentry{affix}{
  name={Affix},
  description={Securely attach or fasten.}
}

\newglossaryentry{air blower}{
  name={Air Blower},
  description={A device that produces a stream of air under pressure.}
}

\newglossaryentry{Assembly Line}{
  name={Assembly Line},
  description={A series of workers and equipment in a factory, assembling a product.}
}

\newglossaryentry{Arduino}{
  name={Arduino},
  description={Arduino is an open-source electronics platform used for creating interactive electronic projects.
}
}

\newglossaryentry{Areal density}{
  name={Areal density},
  description={mass per unit area}
}

\newglossaryentry{Automated Readability Index}{
  name={Automated Readability Index},
  description={A readability formula that calculates the understandability of a text based on the number of characters, words, and sentences.}
}

\newglossaryentry{Baking soda}{
  name={Baking soda},
  description={Alkaline powder for odor absorption and mild stain removal.}
}

\newglossaryentry{Bleaching Powder}{
  name={Bleaching Powder},
  description={A chemical compound used for bleaching and disinfection, often containing calcium hypochlorite.}
}

\newglossaryentry{bone-dry}{
  name={Bone-Dry},
  description={Completely dry, with no moisture remaining.}
}

\newglossaryentry{CPCB}{
  name={CPCB (Central Pollution Control Board)},
  description={A regulatory body in India that sets standards for environmental pollution control.}
}

\newglossaryentry{CaCO3}{
  name={CaCO3},
  description={Calcium carbonate, a common substance that can contribute to water hardness.}
}

\newglossaryentry{Carcinogen}{
  name={Carcinogen},
  description={potential to cause cancer.}
}

\newglossaryentry{Circular Scrubber}{
  name={Circular Scrubber},
  description={A device with a circular motion for scrubbing or cleaning.}
}

\newglossaryentry{chlorinated}{
  name={chlorinated},
  description={state of having chlorine}
}

\newglossaryentry{Coleman Liau Readability Index}{
  name={Coleman Liau Readability Index},
  description={A readability test designed to gauge the understandability of English texts by the average reader.}
}

\newglossaryentry{Compatibility}{
  name={Compatibility},
  description={The ability of the system to work with other components.}
}

\newglossaryentry{Compliance}{
  name={Compliance},
  description={Adherence to rules, regulations, or standards.}
}

\newglossaryentry{Conventional Norms}{
  name={Conventional Norms},
  description={Established standards or practices in the industry.}
}

\newglossaryentry{Contaminants}{
  name={Contaminants},
  description={Unwanted impurities or substances.}
}

\newglossaryentry{Controller}{
  name={Controller},
  description={ In this context, a controller is a device that regulates or manages the operation of the dryer. It could involve setting and maintaining specific conditions such as temperature and drying time.}
}

\newglossaryentry{Control Mechanism}{
  name={Control Mechanism},
  description={ The method used to regulate or manipulate the operation of the system.}
}

\newglossaryentry{Denier}{
  name={Denier},
  description={A unit of measurement for the linear mass density of fibers. It is the mass in grams per 9000 meters of the fiber.}
}

\newglossaryentry{Desizing}{
  name={Desizing},
  description={The removal of sizing agents, such as starch or other chemicals, from fabrics to prepare them for further processing.}
}

\newglossaryentry{Detergent Pretreatment}{
  name={Detergent Pretreatment},
  description={Applying detergent before the main cleaning process.}
}

\newglossaryentry{Detergents}{
  name={Detergents},
  description={Surfactants that remove dirt and stains.}
}

\newglossaryentry{Dye stains}{
  name={Dye stains},
  description={Pigments used in coloring fabrics.}
}

\newglossaryentry{Efficiency}{
  name={Efficiency},
  description={The effectiveness of the system in converting input power to useful output power.}
}

\newglossaryentry{Fabric Stains}{
  name={Fabric Stains},
  description={Unwanted discolorations or marks on textiles.}
}

\newglossaryentry{filleted}{
  name={Filleted},
  description={Somewhat stuck to the side or attached to side walls.}
}

\newglossaryentry{fixtures}{
  name={Fixtures},
  description={a fixed frame.}
}

\newglossaryentry{Flesch Kincaid Grade Level}{
  name={Flesch Kincaid Grade Level},
  description={A readability test that indicates the reading grade level required to understand a particular text.}
}

\newglossaryentry{Flesch Reading Ease}{
  name={Flesch Reading Ease},
  description={A measure of how easy a text is to read; higher scores indicate easier readability.}
}

\newglossaryentry{Gantt Chart}{
  name={Gantt Chart},
  description={A visual representation of a project schedule that shows the start and finish dates of various elements of the project.}
}

\newglossaryentry{Graywater}{
  name={graywater},
  description={Domestic wastewater that does not contain fecal matter, often reused for irrigation or other non-potable purposes.}
}

\newglossaryentry{Grease}{
  name={Grease},
  description={A semiliquid substance used as a lubricant.}
}

\newglossaryentry{Gunning Fog Readability}{
  name={Gunning Fog Readability},
  description={A readability index that estimates the years of formal education a person needs to understand the text on the first reading.}
}

\newglossaryentry{High-pressure jet machine}{
  name={High-pressure jet machine},
  description={A high-pressure reciprocating plunger pump that uses high-pressure water jet cleaning machine to remove mold, grime, dust, mud, and dirt from surfaces and objects such as buildings, vehicles, and concrete road surfaces.}
}

\newglossaryentry{High-speed fan}{
  name={High-speed fan},
  description={A fan capable of generating a fast stream of air used in conjunction with a pipe to create a similar effect as the air blower.}
}

\newglossaryentry{hot air-drying}{
  name={Hot Air-Drying Method},
  description={A drying technique that involves the use of warm or hot air to accelerate the evaporation of moisture.}
}

\newglossaryentry{Hot water washing}{
  name={Hot water washing},
  description={Effective for certain stains (check fabric care label).}
}

\newglossaryentry{Isometric View}{
  name={Isometric View},
  description={A three-dimensional representation of an object, showing all three spatial dimensions in one view.}
}

\newglossaryentry{Lifespan}{
  name={Lifespan},
  description={The expected duration of the washing machine's functional existence.}
}

\newglossaryentry{Logistical Requirements}{
  name={Logistical Requirements},
  description={Specific needs related to the organization and coordination of resources for the washing machine.}
}

\newglossaryentry{Mercerizing}{
  name={Mercerizing},
  description={A textile finishing process that increases the luster and strength of fabric, typically cotton, by treating it with a caustic soda solution.}
}

\newglossaryentry{Meticulous}{
  name={Meticulous},
  description={Very careful and precise.}
}

\newglossaryentry{Network Chart}{
  name={Network Chart},
  description={A visual representation of project tasks and their dependencies in a network or flowchart format.}
}

\newglossaryentry{Nozzle}{
  name={Nozzle},
  description={A device for controlling the direction or flow of a fluid.}
}

\newglossaryentry{nozzle mechanism}{
  name={Nozzle mechanism},
  description={The nozzle and flapper mechanism are a displacement type detector which converts mechanical movement into a pressure signal by covering the opening of a nozzle with a flat plate called the flapper. This restricts fluid flow through the nozzle and generates a pressure signal.}
}

\newglossaryentry{Oil-based stains}{
  name={Oil-based stains},
  description={Grease, wax, lubricants.}
}

\newglossaryentry{Petroleum-based cleaning agents}{
  name={Petroleum-based cleaning agents},
  description={Solvents effective for oil-based stains (e.g., mineral spirits, naphtha).}
}

\newglossaryentry{Pretreatment}{
  name={Pretreatment},
  description={Applying cleaning agent before washing.}
}

\newglossaryentry{Residues}{
  name={Residues},
  description={Any remaining substances or particles.}
}

\newglossaryentry{Resource Breakdown}{
  name={Resource Breakdown},
  description={A breakdown of resources required for a project, often detailing labor, equipment, and materials.}
}

\newglossaryentry{Robust Scrubbing}{
  name={Robust Scrubbing},
  description={Strong, sturdy, and effective scrubbing.}
}

\newglossaryentry{Rough Platform}{
  name={Rough Platform},
  description={A surface with an uneven texture or surface.}
}

\newglossaryentry{Rust stains}{
  name={Rust stains},
  description={Formed by oxidation of iron.}
}

\newglossaryentry{Scour}{
  name={Scour},
  description={To clean or rub using a stiff brush or abrasive.}
}

\newglossaryentry{Scouring}{
  name={Scouring},
  description={The process of cleaning or scrubbing the fabric, often involving the use of detergents or solvents.}
}

\newglossaryentry{Scrubbing Process}{
  name={Scrubbing Process},
  description={The action of cleaning or scrubbing thoroughly.}
}

\newglossaryentry{Setup Time}{
  name={Setup Time},
  description={The time required to prepare the washing machine for operation.}
}

\newglossaryentry{Singeing}{
  name={Singeing},
  description={The process of burning off protruding fibers or impurities from the surface of a fabric, often using a flame or hot surface.}
}

\newglossaryentry{Soap}{
  name={Soap},
  description={A cleansing agent that removes dirt and stains.}
}

\newglossaryentry{Soap solution}{
  name={Soap solution},
  description={A mixture of soap and water used for cleaning purposes.}
}

\newglossaryentry{Sustainability}{
  name={Sustainability},
  description={The ability to maintain or support ecological balance over the long term, including considerations for energy consumption and material choices.}
}

\newglossaryentry{Syrup}{
  name={Syrup},
  description={In the context of the report, it may refer to a concentrated solution of a sugar or sugar substitute in water.}
}

\newglossaryentry{Tailored}{
  name={Tailored},
  description={Customized or adjusted to meet specific requirements.}
}

\newglossaryentry{Tannin}{
  name={Tannin},
  description={A bitter, astringent substance found in plants.}
}

\newglossaryentry{testing phase}{
  name={Testing Phase},
  description={A stage in the development process where the functionality and performance are evaluated.}
}

\newglossaryentry{Thermoplastic Polymer}{
  name={Thermoplastic Polymer},
  description={A type of polymer that becomes pliable or moldable when heated and solidifies upon cooling.}
}

\newglossaryentry{thread count}{
  name={Thread Count},
  description={The number of threads woven together per square inch in a fabric. A higher thread count is generally associated with a finer and more luxurious fabric.}
}

% \newglossaryentry{thread count}{
%   name={Thread Count},
%   description={The number of threads per square inch in a woven fabric, often used as a measure of fabric quality.}
% }

\newglossaryentry{traces}{
  name={Traces},
  description={Small amounts or remnants.}
}

\newglossaryentry{turpentine}{
  name={Turpentine},
  description={A solvent used for thinning and cleaning paint and varnish removal.}
}

% \newglossaryentry{Turpentine}{
%   name={Turpentine},
%   description={Solvent for paint }
% }

\newglossaryentry{user-centric}{
  name={User-Centric},
  description={Designed with a primary focus on meeting the needs and preferences of users.}
}

\newglossaryentry{user-centric washing machine}{
  name={User-centric washing machine},
  description={One of the fundamental aspects of this concept is design, a process into which the needs and}
}

\newglossaryentry{graywater}{
  name={Graywater},
  description={Domestic wastewater that does not contain fecal matter, often reused for irrigation or other non-potable purposes.}
}

\newglossaryentry{guidingframe}{
  name={Guiding Frame},
  description={A frame used to fold the cloth in half vertically.}
}

\newglossaryentry{hcsr04}{
  name={HCSR04 Ultrasonic Sensor},
  description={A specific ultrasonic sensor used in the prototype to detect objects in front of the dispenser.}
}

\newglossaryentry{infraredheating}{
  name={Infrared-based Heating},
  description={A method of heating using infrared radiation. Infrared radiation is a type of electromagnetic radiation that heats objects directly without heating the surrounding air. It's commonly used in appliances like heaters and dryers.}
}

\newglossaryentry{isometricview}{
  name={Isometric View},
  description={A three-dimensional representation of an object, showing all three spatial dimensions in one view.}
}

\newglossaryentry{kw}{
  name={KW (Kilowatts)},
  description={Kilowatts are a unit of power.}
}

\newglossaryentry{lifespan}{
  name={Lifespan},
  description={The duration for which the system is expected to operate.}
}

\newglossaryentry{limitingelementvoltage}{
  name={Limiting Element Voltage},
  description={The maximum voltage the potentiometer can handle.}
}

\newglossaryentry{Lipase}{
  name={Lipase},
  description={An enzyme the body uses to break down fats.}
}

\newglossaryentry{mathmodeling}{
  name={Mathematical Modeling and Simulations},
  description={The use of mathematical equations and computer simulations to analyze and predict the behavior of the system.}
}

\newglossaryentry{maxcurrrating}{
  name={Maximum Current Rating},
  description={Indicates the maximum current the device can draw from the power source.}
}

\newglossaryentry{maxcurrratingperchannel}{
  name={Maximum Current Rating (per channel)},
  description={Indicates the maximum amount of electrical current that can flow through each channel without causing damage. The limit is set at 600 milliamperes (mA).}
}

\newglossaryentry{maxoperatingvoltage}{
  name={Maximum Operating Voltage},
  description={The highest voltage at which the roller or the associated system can safely operate.}
}

\newglossaryentry{mercerizing}{
  name={Mercerizing},
  description={A textile finishing process that increases the luster and strength of fabric, typically cotton, by treating it with a caustic soda solution.}
}

\newglossaryentry{nozzle}{
  name={Nozzle},
  description={A device for controlling the direction or flow of a fluid.}
}

\newglossaryentry{nozzlemechanism}{
  name={Nozzle Mechanism},
  description={The nozzle and flapper mechanism are a displacement type detector that converts mechanical movement into a pressure signal by covering the opening of a nozzle with a flat plate called the flapper. This restricts fluid flow through the nozzle and generates a pressure signal.}
}

\newglossaryentry{petroleumcleaning}{
  name={Petroleum-based Cleaning Agents},
  description={Solvents effective for oil-based stains (e.g., mineral spirits, naphtha).}
}

\newglossaryentry{pretreatment}{
  name={Pretreatment},
  description={Applying cleansing agent before washing.}
}

\newglossaryentry{Prototype}{
  name={Prototype},
  description={An initial version or model of a product that is used to test and develop the design.}
}

\newglossaryentry{resourcebreakdown}{
  name={Resource Breakdown},
  description={A breakdown of resources required for a project, often detailing labor, equipment, and materials.}
}

\newglossaryentry{robustscrubbing}{
  name={Robust Scrubbing},
  description={Strong, sturdy, and effective scrubbing.}
}

\newglossaryentry{ruststains}{
  name={Rust Stains},
  description={Formed by oxidation of iron.}
}

\newglossaryentry{scour}{
  name={Scour},
  description={To clean or rub using a stiff brush or abrasive.}
}

\newglossaryentry{setuptime}{
  name={Setup Time},
  description={The time required to prepare the washing machine for operation.}
}

\newglossaryentry{singeing}{
  name={Singeing},
  description={The process of burning off protruding fibers or impurities from the surface of a fabric, often using a flame or hot surface.}
}

\newglossaryentry{solvent}{
  name={Solvent},
  description={A substance, typically a liquid, capable of dissolving other substances.}
}

\newglossaryentry{spotlifters}{
  name={Spot Lifters},
  description={A unique cleaning aid that removes oil and grease spots and stains from several types of fabrics and clothing.}
}

\newglossaryentry{syrup}{
  name={Syrup},
  description={In the context of the report, it may refer to a concentrated solution of sugar or a sugar substitute in water.}
}

\newglossaryentry{tannin}{
  name={Tannin},
  description={A bitter, astringent substance found in plants.}
}

\newglossaryentry{testingphase}{
  name={Testing Phase},
  description={A stage in the development process where the functionality and performance are evaluated.}
}

\newglossaryentry{thermoplasticpolymer}{
  name={Thermoplastic Polymer},
  description={A type of polymer that becomes pliable or moldable when heated and solidifies upon cooling.}
}

\newglossaryentry{threadcount}{
  name={Thread Count},
  description={The number of threads woven together per square inch in a fabric. A higher thread count is generally associated with a finer and more luxurious fabric.}
}

\newglossaryentry{traces}{
  name={Traces},
  description={Small amounts or remnants.}
}

\newglossaryentry{turpentine}{
  name={Turpentine},
  description={Solvent used for thinning, cleaning paint and varnish removal.}
}

\newglossaryentry{usercentric}{
  name={User-Centric},
  description={Designed with a primary focus on meeting the needs and preferences of users.}
}

\newglossaryentry{usp}{
  name={USP},
  description={Unique Selling Point.}
}

\newglossaryentry{voltage}{
  name={Voltage},
  description={Voltage is a measure of electrical potential difference.}
}

\newglossaryentry{wastetub}{
  name={Waste Tub},
  description={A waste tub at the bottom to collect residual drippings and lint.}
}

\newglossaryentry{watercompartment}{
  name={Water Compartment},
  description={A compartment to store water.}
}


\makeindex
% \usepackage{filecontents}
% \begin{filecontents}{logos/export-data.bib}
% \cite{cleaners_dry_2020}
% \cite{noauthor_dry_nodate}
% \cite{kiron_dry_2021}
% \cite{noauthor_simple_nodate}
% \end{filecontents}
% \addto\captionsenglish{
%   \renewcommand{\tabledematieres}%
%     {Whatever}%
% }
% \renewcommand{\tabledematieres}{Custom Table of Contents Name}
\begin{filecontents}{logos/export-data.bib}
\cite{aqualogic_hot_2023}
\cite{chailoet_analytical_2018}
\cite{cleaners_dry_2020}
\cite{gentlemans_gazette_how_2019}
\cite{green_agitated_nodate}
\end{filecontents}
\titleformat{\subsubsection}[hang]{\hspace{2em}\normalfont\bfseries}{\thesubsubsection}{1em}{}

\title{Tribe D DOC} %Thanks for Rapport CentraleSupelec - Template, By Axel Poupart-Lafarge
% \renewcommand*\contentsname{Tutorials}
% \titleformat{\subsection}[hang]{\hspace{2em}\normalfont\bfseries}{\thesubsection}{1em}
\definecolor{lightergray}{RGB}{230, 230, 230}
\usepackage[backend=biber,style=numeric]{biblatex}
\addbibresource{logos/export-data.bib}
\begin{document}

%----------- Informations du rapport ---------

\logoentreprise{logos/iitd logo.png}
% \logoentreprise{logos/opop1212.png}

% \begin{figure}
%   \centering
%   \includegraphics[width=0.8\linewidth]{https://upload.wikimedia.org/wikipedia/en/thumb/f/fd/Indian_Institute_of_Technology_Delhi_Logo.svg/1200px-Indian_Institute_of_Technology_Delhi_Logo.svg.png}
%   \caption{Your caption here.}
%   \label{fig:your-label}
% \end{figure}
\titre{Tribe D Documentation} % Titre du fichier

\mention{Design and System Laboratory} % Nom de la Mention
\trigrammemention{ELP305} % Pour le bas de la page
\master{Project 1} % Nom du master
%\filiere{Filière Management de projet et Transformation} % Nom de la filière

\eleve{ELP305}

\dates{Week 2 Submission}

% Informations tuteurs écoles
\tuteuruniv{
    \textsc{Prof. Subrat Kar} \\
    % prenom.nom@univ-lemans.fr
} 

\tuteurentreprise{
    
    \textsc{Tribe D (Demantors)} \\
    % \textsc{Coordinators} \\
    % Ayush Sharma (entry no.) \\
    % Ayush Dudawat (entry no.) 
}

%----------- Initialisation -------------------
        
\fairemarges %Afficher les marges
\fairepagedegarde %Créer la page de garde

%----------- Abstract -------------------
% \vspace*{\stretch{1}}
% \begin{center}
% 	\begin{abstract}
%         \lipsum[1]
%         \rule{\linewidth}{0.2 mm} \\[0.4 cm]
%         \begin{center}\textbf{Summary :}\end{center} 
%         \lipsum[2]
%     \end{abstract}
% \end{center}
% \vspace*{\stretch{1}}
\newpage
\section*{Document Navigator}
\begin{itemize}[label=$\bullet$]
  \item \textbf{\hyperref[sec:ourtribeone]{Title}} \hfill \textbf{ \pageref{sec:ourtribeone}}
  \item \textbf{\hyperref[sec:toc]{Body}} \hfill \textbf{ \pageref{sec:toc}}
  \item \textbf{\hyperref[sec:references]{References}} \hfill \textbf{ \pageref{sec:references}}
  \item \textbf{\hyperref[sec:myindex]{Index}} \hfill \textbf{ \pageref{sec:myindex}}
  \item \textbf{\hyperref[sec: glossary]{Glossary}} \hfill \textbf{ \pageref{sec: glossary}}
  \item \textbf{\hyperref[sec:appendix]{Appendices}} \hfill \textbf{ \pageref{sec:appendix}}
  \begin{itemize}[label=$\bullet$]
      \item \hyperref[tab:documentid]{Document ID}
      \hfill\textbf{ \pageref{tab:documentid}}
      \item \hyperref[tab:documentstats]{Document Statistics}
      \hfill\textbf{ \pageref{tab:documentstats}}
      \item \hyperref[tab:readability]{Readability Indices}
      \hfill\textbf{ \pageref{tab:readability}}
  \end{itemize}
  % \begin{itemize}
  %     \item \textbf{\hyperref[sec:documentid]{Document ID}} \hfill \textbf{ \pageref{sec:documentid}}
  %     \item \textbf{\hyperref[sec:documentstats]{Document Statistics}} \hfill \textbf{ \pageref{sec:documentstats}}
  %     \item \textbf{\hyperref[sec:readability]{Readability Indices}} \hfill \textbf{ \pageref{sec:readability}}
  % \end{itemize}
\end{itemize}
\newpage

%------------ Table des matières ----------------

% \tabledematieres
%\tabledematieres % Créer la table de matières

%------------ Corps du rapport ----------------


%------------ Introduction ----------------
\phantomsection
\section*{Week 2 Report: Requirements and Specifications}\label{sec:ourtribeone}
\begin{table}[h]
\centering
  \caption{Team Members list 1}
  \label{tab:team-members}
    \rowcolors{2}{white}{lightergray}
  
  \begin{tabular}{|p{.5cm}|p{3cm}|c|p{2.58cm}|c|p{0.3cm}|}
\hline
SNo. & Name & Roll No. & Position & Email & IF \\
\rowcolor{lightgray}
1 & \href{https://www.linkedin.com/in/ayush-dudawat-6b7a9b222/}{Ayush
Dudawat} & 2021EE10694 & Tribe &
\href{mailto:ee1210694@ee.iitd.ac.in}{\nolinkurl{ee1210694@ee.iitd.ac.in}}
& 1 \\
\rowcolor{lightgray}
&&&Coordinator&& \\
\rowcolor{lightgray}

2 & \href{https://www.linkedin.com/in/ayush-sharma-b01346224/}{Ayush
Sharma} & 2021MT10244 & Tribe  &
\href{mailto:mt1210244@maths.iitd.ac.in}{\nolinkurl{mt1210244@maths.iitd.ac.in}}
& 1 \\
\rowcolor{lightgray}
&&&Coordinator&&\\
\rowcolor{lightergray}

3 & \href{https://www.linkedin.com/in/nitesh-singh-a79a17223/}{Nitesh
Singh} & 2021MT10250 & Documentation   &
\href{mailto:mt1210250@maths.iitd.ac.in}{\nolinkurl{mt1210250@maths.iitd.ac.in}}
& 1 \\
\rowcolor{lightergray}
&&&Coordinator&&\\
\rowcolor{lightergray}

4 & \href{https://www.linkedin.com/in/vansh-jain-36569b225/}{Vansh Jain}
& 2021MT10234 & Research  Coordinator &
\href{mailto:mt1210234@maths.iitd.ac.in}{\nolinkurl{mt1210234@maths.iitd.ac.in}}
& 1 \\
\rowcolor{lightergray}
5 &
\href{https://www.linkedin.com/in/sharesth-thakan-249504250/}{Sharesth
Thakan} & 2021EE30730 & Fabrication and testing  Coordinator &
\href{mailto:ee3210730@ee.iitd.ac.in}{\nolinkurl{ee3210730@ee.iitd.ac.in}}

& 1 \\
\rowcolor{lightergray}

6 & \href{https://www.linkedin.com/in/abhas-porov-b69077248/}{Abhas
Porov} & 2021EE10781 & Electrical and Simulation Coordinator &
\href{mailto:ee1210781@ee.iitd.ac.in}{\nolinkurl{ee1210781@ee.iitd.ac.in}}
& 1 \\
\hline
7 & \href{https://www.linkedin.com/in/tanisha-jangra-5203132ab}{Tanisha}
& 2021MT10927 & Research  &
\href{mailto:mt1210927@maths.iitd.ac.in}{\nolinkurl{mt1210927@maths.iitd.ac.in}}
& 0.6 \\
8 &
\href{https://www.linkedin.com/in/shreyansh-jain-6abb9124b/}{Shreyansh
Jain} & 2021MT10930 & Research  &
\href{mailto:mt1210930@maths.iitd.ac.in}{\nolinkurl{mt1210930@maths.iitd.ac.in}}
& 0.8 \\
9 & \href{https://www.linkedin.com/in/rishika-arya-266082279/}{Rishika
Arya} & 2021MT10926 & Research  &
\href{mailto:mt1210926@maths.iitd.ac.in}{\nolinkurl{mt1210926@maths.iitd.ac.in}}
& 1 \\
10 &
\href{https://www.linkedin.com/in/sarmistha-subhadarshini-507172243}{Sarmistha
} & 2021MT10261 & Research  &
\href{mailto:mt1210261@maths.iitd.ac.in}{\nolinkurl{mt1210261@maths.iitd.ac.in}}
& 1 \\

11 &
\href{https://www.linkedin.com/in/anshika-prajapati-9b855022b/}{Anshika
Prajapati} & 2021MT60961 & Research  &
\href{mailto:mt6210961@maths.iitd.ac.in}{\nolinkurl{mt6210961@maths.iitd.ac.in}}
& 1 \\
12 & \href{https://www.linkedin.com/in/rupam-kumawat-b27949253/}{Rupam Kumawat} & 2021MT60267 & Research &
\href{mailto:mt6210267@maths.iitd.ac.in}{\nolinkurl{mt6210267@maths.iitd.ac.in}}
& 1 \\
 
        % \textbf{S.No.} & \textbf{Name} & \textbf{Roll No.} & \textbf{Position} & \textbf{Email} & \textbf{Group} \\
        % \hline
        13 & \href{www.linkedin.com/in/sakshimagarkar/}{Sakshi Magarkar} & 2021MT60965 & Research & \href{mailto:mt6210965@maths.iitd.ac.in}{\nolinkurl{mt6210965@maths.iitd.ac.in}} & 1 \\
        14 & \href{https://www.linkedin.com/in/aniket-pandey-b5b9a1263/}{Aniket Pandey} & 2021MT60266 & Research & \href{mailto:mt6210266@maths.iitd.ac.in}{\nolinkurl{mt6210266@maths.iitd.ac.in}} & 1 \\
        15 & \href{https://www.linkedin.com/in/nancy-kansal-1b5384234/}{Nancy Kansal} & 2021MT10905 & Research & \href{mailto:mt1210905@maths.iitd.ac.in}{\nolinkurl{mt1210905@maths.iitd.ac.in}} & 1 \\
        16 & \href{https://www.linkedin.com/in/divyansh-agarwal-22989525b/}{Diyvansh Agarwal} & 2021EE10035 & Research & \href{mailto:ee1210035@ee.iitd.ac.in}{\nolinkurl{ee1210035@ee.iitd.ac.in}} & 0.9 \\
        17 & \href{https://www.linkedin.com/in/mukund-aggarwal}{Mukund Aggarwal} & 2021MT60939 & Research & \href{mailto:mt6210939@maths.iitd.ac.in}{\nolinkurl{mt6210939@maths.iitd.ac.in}} & 1 \\
        \hline
\end{tabular}

\end{table}
\begin{table}[h]\label{sec:ourtribetwo}
\centering
  \caption{Team Members list 2}
  % \label{tab:team-members}
    \rowcolors{2}{white}{lightergray}
  \begin{tabular}{|p{.3cm}|p{2.9cm}|c|p{2cm}|c|p{.3cm}|}
  \hline
18 & \href{https://www.linkedin.com/in/tanishk-singh-80ba09224/}{Tanishk Singh} & 2021EE10167 & Research & \href{mailto:ee1210167@ee.iitd.ac.in}{\nolinkurl{ee1210167@ee.iitd.ac.in}} & 0.6 \\
        19 & \href{https://www.linkedin.com/in/akshansh-rajora-5794b5228}{Akshansh Rajora} & 2021MT10933 & Research & \href{mailto:mt1210933@maths.iitd.ac.in}{\nolinkurl{mt1210933@maths.iitd.ac.in}} & 0.6 \\
        20 & \href{https://www.linkedin.com/in/ayush-madhur-40a575236/}{ Ayush Madhur}& 2021EE10161 & Research & \href{mailto:ee1210161@ee.iitd.ac.in}{\nolinkurl{ee1210161@ee.iitd.ac.in}} & 0.6 \\
        21 &\href{https://www.linkedin.com/in/keshvi-tomer-4b0331236/}{Keshvi Tomar} & 2021EE10682 & Research & \href{mailto:ee1210682@ee.iitd.ac.in}{\nolinkurl{ee1210682@ee.iitd.ac.in}} & 0.9 \\
        22 & \href{https://www.linkedin.com/in/kanak-kumar-538ab2247/}{Kanak Kuma} Singh} & 2021EE10163 & Research & \href{mailto:ee1210163@ee.iitd.ac.in}{\nolinkurl{ee1210163@ee.iitd.ac.in}} & 0.6 \\
        23 & \href{https://www.linkedin.com/in/aravind-udupa-266a52223/}{Aravind Udupa} & 2021MT60940 & Research & \href{mailto:mt6210940@maths.iitd.ac.in}{\nolinkurl{mt6210940@maths.iitd.ac.in}} & 1 \\
        24 & \href{https://www.linkedin.com/in/arpit-rathore-56b535223/}{Arpit Rathore} & 2021MT10920 & Research & \href{mailto:mt1210920@maths.iitd.ac.in}{\nolinkurl{mt1210920@maths.iitd.ac.in}} & 1 \\
        \hline
        25 & \href{https://www.linkedin.com/in/vandit-srivastava}{Vandit Srivastava} & 2021EE10640 & Electrical & \href{mailto:ee1210640@ee.iitd.ac.in}{\nolinkurl{ee1210640@ee.iitd.ac.in}} & 1 \\
        26 & \href{https://www.linkedin.com/in/ankita-meena-2b919a236/}{Ankita Meena} & 2021EE10173 & Electrical & \href{mailto:ee1210173@ee.iitd.ac.in}{\nolinkurl{ee1210173@ee.iitd.ac.in}} & 1 \\
        27 & \href{https://www.linkedin.com/in/aditya-gupta-178638228}{Aditya Gupta} & 2021EE30713 & Electrical & \href{mailto:ee3210713@ee.iitd.ac.in}{\nolinkurl{ee3210713@ee.iitd.ac.in}} & 1 \\
        28 & \href{https://www.linkedin.com/in/aditya-bhalotia-756654253}{Aditya Bhalotia} & 2021EE30698 & Electrical & \href{mailto:ee3210698@ee.iitd.ac.in}{\nolinkurl{ee3210698@ee.iitd.ac.in}} & 1 \\
        29 & \href{https://www.linkedin.com/in/ayush-shrivastava-264398248}{Ayush Shrivastava} & 2021EE10632 & Electrical & \href{mailto:ee1210632@ee.iitd.ac.in}{\nolinkurl{ee1210632@ee.iitd.ac.in}} & 1 \\
        30 & \href{https://www.linkedin.com/in/harshit-nagar-178a33253}{Harshit Nagar} & 2021EE10155 & Electrical & \href{mailto:ee1210155@ee.iitd.ac.in}{\nolinkurl{ee1210155@ee.iitd.ac.in}} & 1 \\
        
        31 & \href{www.linkedin.com/in/shreyansh-jaiswal-4b79b2228}{Shreyansh Jaiswal} & 2021EE10154 & Electrical & \href{mailto:ee1210154@ee.iitd.ac.in}{\nolinkurl{ee1210154@ee.iitd.ac.in}} & 1 \\
        32 & \href{https://www.linkedin.com/in/akshar-tripathi-9a267425b/}{Akshar Tripathi} & 2021EE10980 & Electrical & \href{mailto:ee1210980@ee.iitd.ac.in}{\nolinkurl{ee1210980@ee.iitd.ac.in}} & 1 \\
        33 & \href{https://www.linkedin.com/in/muskan-yadav-2b0651b4}{Muskan Yadav} & 2021EE10686 & Electrical & \href{mailto:ee1210686@ee.iitd.ac.in}{\nolinkurl{ee1210686@ee.iitd.ac.in}} & 1 \\
        34 & \href{https://www.linkedin.com/in/pavan-bharadwaj-07025a281/}{Pavan Bharadwaj} & 2021EE10630 & Electrical & \href{mailto:ee1210630@ee.iitd.ac.in}{\nolinkurl{ee1210630@ee.iitd.ac.in}} & 1 \\
        35 & \href{https://www.linkedin.com/in/mokshavi-reddy-93b41a255}{Aluka Mokshavi} & 2021MT10909 & Electrical & \href{mailto:mt1210909@maths.iitd.ac.in}{\nolinkurl{mt1210909@maths.iitd.ac.in}} & 1 \\
        36 & \href{https://www.linkedin.com/in/sathvika-palle-28a13025a}{Palle Sathvika} & 2021MT10928 & Electrical & \href{mailto:mt1210928@maths.iitd.ac.in}{\nolinkurl{mt1210928@maths.iitd.ac.in}} & 1 \\
        37 & \href{www.linkedin.com/in/shubham-anand-055423252}{Shubham Anand} & 2021EE10674 & Electrical & \href{mailto:ee1210674@ee.iitd.ac.in}{\nolinkurl{ee1210674@ee.iitd.ac.in}} & 1 \\
        38 & \href{https://www.linkedin.com/in/sanu-a5b6a72ab/}{Kumar Sanu Singh }& 2021EE31213 & Electrical & \href{mailto:ee3211213@ee.iitd.ac.in}{\nolinkurl{ee3211213@ee.iitd.ac.in}} & 1 \\
        \hline
        39 & \href{https://www.linkedin.com/in/rahul-kumar-9a021a236/}{Rahul Kumar} & 2021MT10893 & Fabrication & \href{mailto:mt1210893@maths.iitd.ac.in}{\nolinkurl{mt1210893@maths.iitd.ac.in}} & 1 \\
        40 & \href{https://www.linkedin.com/in/manav-garg-0a240a175}{Manav Garg} & 2021EE30017 & Fabrication & \href{mailto:ee3210017@ee.iitd.ac.in}{\nolinkurl{ee3210017@ee.iitd.ac.in}} & 1 \\
        41 & \href{www.linkedin.com/in/kushagrgoyal}{Kushagr Goyal} & 2021EE10634 & Fabrication & \href{mailto:ee1210634@ee.iitd.ac.in}{\nolinkurl{ee1210634@ee.iitd.ac.in}} & 1 \\
        42 & \href{https://www.linkedin.com/in/champak-swargiary-a87b04230/}{Champak Swargiary} & 2021MT10263 & Fabrication & \href{mailto:mt1210263@maths.iitd.ac.in}{\nolinkurl{mt1210263@maths.iitd.ac.in}} & 1 \\
        
43 & \href{https://www.linkedin.com/in/ajay-ramavath-/}{Ajay Naik} &
2020MT60888 & Fabrication  &
\href{mailto:mt6210888@maths.iitd.ac.in}{\nolinkurl{mt6210888@maths.iitd.ac.in}}
& 0.5 \\
44 & \href{https://www.linkedin.com/in/aryan-sharma-326657230/}{Aryan
Sharma} & 2021EE10141 & Fabrication  &
\href{mailto:ee1210141@ee.iitd.ac.in}{\nolinkurl{ee1210141@ee.iitd.ac.in}}
& 0.5 \\
\hline
\end{tabular}

\end{table}
\begin{table}\label{sec:ourtribethree}
% \centering
  \caption{Team Members list 3}
  \label{tab:team-members}
    \rowcolors{2}{white}{lightergray}
  
  \begin{tabular}{|p{.3cm}|p{3cm}|c|p{2.7cm}|c|c|}
  \hline
45 & \href{https://www.linkedin.com/in/vadlapudi-manoj-5a764825a/}{Vadlapudi Manoj} &
2021MT10245 & Documentation  &
\href{mailto:mt1210245@maths.iitd.ac.in}{\nolinkurl{mt1210245@maths.iitd.ac.in}}
& 1 \\
46 & \href{https://www.linkedin.com/in/bhavik-garg-4b214422a}{Bhavik
Garg} & 2021EE10657 & Documentation  &
\href{mailto:ee1210657@ee.iitd.ac.in}{\nolinkurl{ee1210657@ee.iitd.ac.in}}
& 1 \\
47 & \href{https://www.linkedin.com/in/ishu-ishu-9241242ab/}{Ishu} &
2021EE30735 & Documentation  &
\href{mailto:ee3210735@ee.iitd.ac.in}{\nolinkurl{ee3210735@ee.iitd.ac.in}}
& 1 \\
48 &
\href{https://www.linkedin.com/in/sashidhar-alvakonda-32b9011a5}{Alvakonda
Sashidhar} & 2021EE30744 & Documentation  &
\href{mailto:ee3210744@ee.iitd.ac.in}{\nolinkurl{ee3210744@ee.iitd.ac.in}}
& 1 \\
49 &
\href{https://www.linkedin.com/in/harshdeep-shakya-507304236/}{Harshdeep
Shakya} & 2021EE30745 & Documentation  &
\href{mailto:ee3210745@ee.iitd.ac.in}{\nolinkurl{ee3210745@ee.iitd.ac.in}}
& 1 \\
50 &
\href{https://www.linkedin.com/in/abhinava-a-mohanty-30a3a6232}{Abhinava
Anwesha Mohanty} & 2021EE10136 & Documentation  &
\href{mailto:ee1210136@ee.iitd.ac.in}{\nolinkurl{ee1210136@ee.iitd.ac.in}}
& 1 \\

51 & \href{www.linkedin.com/in/atishay-aggarwal-066414226}{Atishay
Aggarwal} & 2021MT60941 & Documentation  &
\href{mailto:mt6210941@maths.iitd.ac.in}{\nolinkurl{mt6210941@maths.iitd.ac.in}}
& 1 \\
52 & \href{https://www.linkedin.com/in/srinath-k-s-875834222/}{Srinath K
S} & 2021MT10912 & Documentation  &
\href{mailto:mt1210912@maths.iitd.ac.in}{\nolinkurl{mt1210912@maths.iitd.ac.in}}
& 1 \\
53 & \href{https://www.linkedin.com/in/kshitij-kumar-gautam/}{Kshitij K
Gautam} & 2021MT60269 & Documentation  &
\href{mailto:mt6210269@maths.iitd.ac.in}{\nolinkurl{mt6210269@maths.iitd.ac.in}}
& 1 \\
54 & \href{https://www.linkedin.com/in/chandan-kumar-774813224}{Chandan
Kumar} & 2021MT60268 & Documentation  &
\href{mailto:mt6210268@maths.iitd.ac.in}{\nolinkurl{mt6210268@maths.iitd.ac.in}}
& 1 \\
55 & \href{https://www.linkedin.com/in/naunidh-singh-0b256a22b/}{Naunidh
Singh} & 2021MT60956 & Documentation  &
\href{mailto:mt6210956@maths.iitd.ac.in}{\nolinkurl{mt6210956@maths.iitd.ac.in}}
& 1 \\
56 & \href{www.linkedin.com/in/vipul-yadav-6142a6287}{Vipul} &
2021EE30731 & Documentation  &
\href{mailto:ee3210731@ee.iitd.ac.in}{\nolinkurl{ee3210731@ee.iitd.ac.in}}
& 1 \\
57 & \href{https://www.linkedin.com/in/amit-singh-221888236/}{Amit
Singh} & 2021MT10921 & Documentation  &
\href{mailto:mt1210921@maths.iitd.ac.in}{\nolinkurl{mt1210921@maths.iitd.ac.in}}
& 1 \\
58 &
\href{https://www.linkedin.com/in/sumanth-mandala-868a1a2aa/}{Sumanth
Mandala} & 2021EE10153 & Documentation  &
\href{mailto:ee1210153@ee.iitd.ac.in}{\nolinkurl{ee1210153@ee.iitd.ac.in}}
& 1 \\
59 & \href{https://www.linkedin.com/in/prabhat-babu-490096282}{Prabhat
Babu} & 2021MT10255 & Documentation  &
\href{mailto:mt1210255@maths.iitd.ac.in}{\nolinkurl{mt1210255@maths.iitd.ac.in}}
& 1 \\

\hline
\end{tabular}

\end{table}
% \newpage
\section*{List of People with IF less than 1}\label{sec:IFlessthan1}
% Everyone has got IF equal to 1 in this report
\begin{table}[h]
% \centering
  \caption{Team Members with IF <1}
  \label{tab:team-memberswithiflessthan}
    \rowcolors{2}{white}{lightergray}
  
  \begin{tabular}{|p{.3cm}|p{3cm}|c|p{2.7cm}|c|c|}
  \hline
1 & \href{https://www.linkedin.com/in/tanisha-jangra-5203132ab}{Tanisha}
& 2021MT10927 & Research  &
\href{mailto:mt1210927@maths.iitd.ac.in}{\nolinkurl{mt1210927@maths.iitd.ac.in}}
& 0.6 \\
2 &
\href{https://www.linkedin.com/in/shreyansh-jain-6abb9124b/}{Shreyansh
Jain} & 2021MT10930 & Research  &
\href{mailto:mt1210930@maths.iitd.ac.in}{\nolinkurl{mt1210930@maths.iitd.ac.in}}
& 0.8 \\
3 & \href{https://www.linkedin.com/in/divyansh-agarwal-22989525b/}{Diyvansh Agarwal} & 2021EE10035 & Research & \href{mailto:ee1210035@ee.iitd.ac.in}{\nolinkurl{ee1210035@ee.iitd.ac.in}} & 0.9 \\
4 & \href{https://www.linkedin.com/in/tanishk-singh-80ba09224/}{Tanishk Singh} & 2021EE10167 & Research & \href{mailto:ee1210167@ee.iitd.ac.in}{\nolinkurl{ee1210167@ee.iitd.ac.in}} & 0.6 \\
5 & \href{https://www.linkedin.com/in/akshansh-rajora-5794b5228}{Akshansh Rajora} & 2021MT10933 & Research & \href{mailto:mt1210933@maths.iitd.ac.in}{\nolinkurl{mt1210933@maths.iitd.ac.in}} & 0.6 \\
6 & \href{https://www.linkedin.com/in/ayush-madhur-40a575236/}{Ayush Madhur} & 2021EE10161 & Research & \href{mailto:ee1210161@ee.iitd.ac.in}{\nolinkurl{ee1210161@ee.iitd.ac.in}} & 0.6 \\

7 & \href{https://www.linkedin.com/in/keshvi-tomer-4b0331236/}{Keshvi Tomar} & 2021EE10682 & Research & \href{mailto:ee1210682@ee.iitd.ac.in}{\nolinkurl{ee1210682@ee.iitd.ac.in}} & 0.9 \\
8 & \href{https://www.linkedin.com/in/kanak-kumar-538ab2247/}{Kanak Kumar Singh} & 2021EE10163 & Research & \href{mailto:ee1210163@ee.iitd.ac.in}{\nolinkurl{ee1210163@ee.iitd.ac.in}} & 0.6 \\ 
9 & \href{https://www.linkedin.com/in/ajay-ramavath-/}{Ajay Naik} &
2020MT60888 & Fabrication  &
\href{mailto:mt6210888@maths.iitd.ac.in}{\nolinkurl{mt6210888@maths.iitd.ac.in}}
& 0.5 \\
10 & \href{https://www.linkedin.com/in/aryan-sharma-326657230/}{Aryan
Sharma} & 2021EE10141 & Fabrication  &
\href{mailto:ee1210141@ee.iitd.ac.in}{\nolinkurl{ee1210141@ee.iitd.ac.in}}
& 0.5 \\
% 54 & \href{https://www.linkedin.com/in/chandan-kumar-774813224}{Chandan
% Kumar} & 2021MT60268 & Documentation  &
% \href{mailto:mt6210268@maths.iitd.ac.in}{\nolinkurl{mt6210268@maths.iitd.ac.in}}
% & 1 \\
% 55 & \href{https://www.linkedin.com/in/naunidh-singh-0b256a22b/}{Naunidh
% Singh} & 2021MT60956 & Documentation  &
% \href{mailto:mt6210956@maths.iitd.ac.in}{\nolinkurl{mt6210956@maths.iitd.ac.in}}
% & 1 \\
% 56 & \href{www.linkedin.com/in/vipul-yadav-6142a6287}{Vipul} &
% 2021EE30731 & Documentation  &
% \href{mailto:ee3210731@ee.iitd.ac.in}{\nolinkurl{ee3210731@ee.iitd.ac.in}}
% & 1 \\
% 57 & \href{https://www.linkedin.com/in/amit-singh-221888236/}{Amit
% Singh} & 2021MT10921 & Documentation  &
% \href{mailto:mt1210921@maths.iitd.ac.in}{\nolinkurl{mt1210921@maths.iitd.ac.in}}
% & 1 \\
% 58 &
% \href{https://www.linkedin.com/in/sumanth-mandala-868a1a2aa/}{Sumanth
% Mandala} & 2021EE10153 & Documentation  &
% \href{mailto:ee1210153@ee.iitd.ac.in}{\nolinkurl{ee1210153@ee.iitd.ac.in}}
% & 1 \\
% 59 & \href{https://www.linkedin.com/in/prabhat-babu-490096282}{Prabhat
% Babu} & 2021MT10255 & Documentation  &
% \href{mailto:mt1210255@maths.iitd.ac.in}{\nolinkurl{mt1210255@maths.iitd.ac.in}}
% & 1 \\

\hline
\end{tabular}

\end{table}

\textbf{Reason:} Assigned tasks were not completed, Low participation in most of the meetings even after multiple reminders on the group. No inputs were given for the research stage. No role in CAD model designing
\newpage
\renewcommand{\contentsname}{Table of contents}\phantomsection\label{sec:toc}
\toc
% \section*{Table Of Content}\phantomsection\label{sec:toc}
% \begin{enumerate}
%   % \item \textbf{\hyperref[sec:ourtribe]{Team Members}} \hfill \textbf{ \pageref{sec:ourtribe}}
%   % \item \textbf{\hyperref[sec:IFlessthan1]{Team Members whose Involvement Factor is less than 1}} \hfill \textbf{ \pageref{sec:IFlessthan1}}
%   \item \textbf{\hyperref[sec:listoftables]{List of Tables}} \hfill \textbf{ \pageref{sec:listoftables}}
%   \item \textbf{\hyperref[sec:listoffigures]{List of Figures}} \hfill \textbf{ \pageref{sec:listoffigures}}
%   \item \textbf{\hyperref[sec:abbrevations]{Abbrevations}} \hfill \textbf{\pageref{sec:abbrevations}}
%   \item \textbf{\hyperref[sec: indexredirect]{Index}} \hfill \textbf{\pageref{sec: indexredirect}}
%   \item \textbf{\hyperref[sec: glossaryredirect]{Glossary}} \hfill \textbf{\pageref{sec: glossaryredirect}}
%   \item \textbf{\hyperref[sec:mindmap]{MindMap}} \hfill \textbf{\pageref{sec:mindmap}}
%   \item \textbf{\hyperref[sec:projectmanagement]{Project Management}} \hfill \textbf{\pageref{sec:projectmanagement}}
%   \item \textbf{\hyperref[sec:abstract]{Abstract}} \hfill \textbf{\pageref{sec:abstract}}
%   \item \textbf{\hyperref[sec:motivation]{Motivation}} \hfill \textbf{\pageref{sec:motivation}}
% \item \textbf{\hyperref[sec:mechanism]{Mechanism}} \hfill \textbf{\pageref{sec:mechanism}}
%   \begin{enumerate}
%     \item {\hyperref[sec:removalofdust]{Removal of dust through air}} \hfill \textbf{\pageref{sec:removalofdust}}
%     \item {\hyperref[sec:soapwater]{Soap Water Mixture}} \hfill \textbf{ \pageref{sec:soapwater}}
%     \item {\hyperref[sec:scrubbing]{Scrubbing}} \hfill \textbf{\pageref{sec:scrubbing}}
%     \item {\hyperref[sec:water]{Water}} \hfill \textbf{\pageref{sec:water}}
%     \item {\hyperref[sec:drying]{Drying}} \hfill \textbf{\pageref{sec:drying}}
%   \end{enumerate}
%   \item \textbf{\hyperref[sec:requirements]{Requirements}} \hfill \textbf{ \pageref{sec:requirements}}
%   \begin{enumerate}
%   %     \item
%   % \end{itemize}
%     \item {\hyperref[sec:inputspec]{Input Specifications}} \hfill \textbf{\pageref{sec:inputspec}}
% \begin{enumerate}
%     \item {\hyperref[sec:matspec]{Material Specifications}} \hfill \textbf{\pageref{sec:matspec}}
%     \item {\hyperref[sec:dimensions]{Dimensions}} \hfill \textbf{\pageref{sec:dimensions}}
%     \item {\hyperref[sec:clothchar]{Cloth Characteristics}} \hfill \textbf{\pageref{sec:clothchar}}
%     \item {\hyperref[sec:cleanlim]{Cleaning Limitations}} \hfill \textbf{\pageref{sec:cleanlim}}
% \end{enumerate}
% %     \item {\hyperref[sec:clientpref]{Client Preferences}} \hfill \textbf{ \pageref{sec:clientpref}}
% %     \begin{enumerate}
% %     \item {\hyperref[sec:cnspref]{Cost and Services Preferences}}
% %     \hfill \textbf{\pageref{sec:cnspref}}
% % \end{enumerate}

%     \item {\hyperref[sec:outputreq]{Output Requirements}} \hfill \textbf{\pageref{sec:outputreq}}
% \begin{enumerate}
%     \item {\hyperref[sec:desout]{Desired Output}} \hfill \textbf{\pageref{sec:desout}}
%     \item {\hyperref[sec:clientresp]{Client Responibilities}} \hfill \textbf{\pageref{sec:clientresp}}
% \end{enumerate}
%     \item {\hyperref[sec:powerreq]{Power Requirements}} \hfill \textbf{\pageref{sec:powerreq}}
% \begin{enumerate}
%     \item {\hyperref[sec: vnpreq]{Voltage and Phase Requirements}}
%     \hfill \textbf{\pageref{sec: vnpreq}}
%     \item {\hyperref[sec: opexp]{Operational Expectations}} 
%     \hfill \textbf{\pageref{sec: opexp}}
% \end{enumerate}
%     \item {\hyperref[sec:logireq]{Logistical requirements}} \hfill \textbf{\pageref{sec:logireq}}
%     \begin{enumerate}
%     \item {\hyperref[sec:machtype]{Machine Type and Features}} \hfill \textbf{\pageref{sec:machtype}}
%     \item {\hyperref[sec:machtype]{Washing Medium Features}} \hfill \textbf{\pageref{sec:wmtype}}
% \end{enumerate}
%      \item {\hyperref[subsec:envireq]{Environmental Requirements}} \hfill \textbf{\pageref{subsec:envireq}}
%      \begin{enumerate}
%     \item {\hyperref[sec:noisecomp]{Noise and compliance}} \hfill \textbf{\pageref{sec:noisecomp}}
%     \item {\hyperref[sec:suspref]{Sustainability Preferences}} \hfill \textbf{\pageref{sec:suspref}}
% \end{enumerate}
%       \item {\hyperref[sec:sitereq]{Site requirements}} \hfill \textbf{\pageref{sec:sitereq}}
%       \begin{enumerate}
%     \item {\hyperref[sec:essenforsite]{Essentials for the site}} \hfill \textbf{\pageref{sec:essenforsite}}
%     \item {\hyperref[sec:watersource]{Water Source and structural considerations}} \hfill \textbf{\pageref{sec:watersource}}
% \end{enumerate}
%        \item {\hyperref[sec:timereq]{Time requirements}} \hfill \textbf{\pageref{sec:timereq}}
%        \begin{enumerate}
%     \item {\hyperref[sec: cnstimes]{Cleaning and setup time}} \hfill \textbf{\pageref{sec: cnstimes}}
%     \item {\hyperref[sec: destimereq]{Design Time Requirements!!!}} \hfill \textbf{\pageref{sec: destimereq}}
%     \item {\hyperref[sec: timemarketreq]{Time to Market Requirements}} \hfill \textbf{\pageref{sec: timemarketreq}}
%     \item {\hyperref[sec: lifetimereq]{LifeTime Requirements}} \hfill \textbf{\pageref{sec: lifetimereq}}
%     \item {\hyperref[sec: endoflifereq]{End of Life Requirements}} \hfill \textbf{\pageref{sec: endoflifereq}}
% %     % \item {\hyperref[sec:outputreq]{Water Source and structural considerations}} \hfill \textbf{\pageref{sec:scrubbing}}
% \end{enumerate}
% %      \item {\hyperref[sec:lifereq]{Lifetime requirements}} \hfill \textbf{\pageref{sec:lifereq}}
% %      \begin{enumerate}
% %     \item {\hyperref[sec:lifespanandservice]{Lifespan and service times}} \hfill \textbf{\pageref{sec:lifespanandservice}}
% %     \item {\hyperref[sec:outputreq]{Water Source and structural considerations}} \hfill \textbf{\pageref{sec:scrubbing}}
% % \end{enumerate}
%       \item {\hyperref[sec:otherreq]{Other non functional requirements}} \hfill \textbf{\pageref{sec:otherreq}}
%       \begin{enumerate}
%     \item {\hyperref[sec:misccons]{Miscellaneous Considerations}} \hfill \textbf{\pageref{sec:misccons}}
% \end{enumerate}
%   \end{enumerate}
%   \item \textbf{\hyperref[sec:compana]{Component Analysis}} \hfill \textbf{ \pageref{sec:compana}}
%   \begin{enumerate}
%     \item {\hyperref[sec:roller]{Roller}} \hfill \textbf{\pageref{sec:roller}}
%   \begin{enumerate}
%     \item {\hyperref[sec:contdcmotor]{Controlling DC motor using adruino}} \hfill \textbf{\pageref{sec:contdcmotor}}
%     \item {\hyperref[sec:adriono]{Arduino\index{Arduino} and L293D Circuit Diagram}} \hfill \textbf{\pageref{sec:adriono}}
% \end{enumerate}
% \end{enumerate}
%   \item \textbf{\hyperref[sec:specs]{Specifications}} \hfill \textbf{ \pageref{sec:specs}}
%   \begin{enumerate}
%     \item {\hyperref[sec:energyspecs]{Energy Specifications}} \hfill \textbf{\pageref{sec:energyspecs}}
%     \item {\hyperref[sec:spacespecs]{Space Specifications}} 
%     \hfill \textbf{\pageref{sec:spacespecs}}
%     \item {\hyperref[sec:powspecs]{Power Specifications}} \hfill \textbf{\pageref{sec:powspecs}}
%     \item {\hyperref[sec:costspecs]{Cost Specifications}} \hfill \textbf{\pageref{sec:costspecs}}
%     \item {\hyperref[sec:perfspecs]{Performance Specifications}} \hfill \textbf{\pageref{sec:perfspecs}}
%     \item {\hyperref[sec:mpspecs]{Man Power Specifications}} \hfill \textbf{\pageref{sec:mpspecs}}
%     \item {\hyperref[sec:msspecs]{Milestone Specifications}} \hfill \textbf{\pageref{sec:msspecs}}
%   \end{enumerate}
%   % \item \textbf{\hyperref[sec:design]{Design}} \hfill \textbf{ \pageref{sec:design}}
%   % \item \textbf{\hyperref[sec:closure]{Closure}} \hfill \textbf{ \pageref{sec:closure}}
%   % \item \textbf{\hyperref[sec:reuse]{Reuse}} \hfill \textbf{ \pageref{sec:reuse}}
  
%   % \item \textbf{\hyperref[sec:references]{References}} \hfill \textbf{ \pageref{sec:references}}
%   % \item \textbf{\hyperref[sec:appendix]{Appendix}} \hfill \textbf{ \pageref{sec:appendix}}
%   %   \begin{itemize}
%   %   \item {\hyperref[sec:documentstats]{Document Statistics}} \hfill \textbf{\pageref{sec:documentstats}}
%   %   \item {\hyperref[sec:readability]{Readability Indices}} \hfill \textbf{ \pageref{sec:readability}}
%   % \end{itemize}
% \end{enumerate}
\clearpage


\section{List of Tables}\label{sec:listoftables} 
% % \ref{_our_tribe}{Our Tribe}
% \begin{enumerate}
%     \item \hyperref[sec:ourtribeone]{Our Tribe Table 1}\hfill \pageref{sec:ourtribeone}
%     \item \hyperref[sec:ourtribetwo]{Our Tribe Table 2}\hfill \pageref{sec:ourtribetwo}
%     \item \hyperref[sec:ourtribethree]{Our Tribe Table 3}\hfill \pageref{sec:ourtribethree}
    
% \end{enumerate}
\renewcommand{\listtablename}{}

\listoftables
\section{List of Figures}\label{sec:listoffigures}
\renewcommand{\listfigurename}{}
\setcounter{figure}{0}
\listoffigures

% \begin{enumerate}
%     \item {\hyperref[sec:figone]{Isometric View Figure 1\index{\gls{Isometric View} Figure 1}}\hfill \pageref{sec:figone}
%     \item \hyperref[sec:figtwo]{Isometric View Figure 2\index{\gls{Isometric View} Figure 2}}\hfill \pageref{sec:figtwo}
%     \item \hyperref[fig:reqmindmap]{Requirements Mind Map}\hfill \pageref{fig:reqmindmap}
%     \item \hyperref[fig:outlinemindmap]{Outline Mind Map}\hfill \pageref{fig:outlinemindmap}
    
% \end{enumerate}



% \ref{sec:soapwater}{Isometric view Figure 1}\\
% \ref{_fig2_isometric}{Isometric view Figure 2} \\
% \ref{_requirements_mind_map}{Mind Map for Requirements}
\newpage
\section{List of Abbreviations}\label{sec:abbrevations}
% \begin{itemize}
   % \text{$\bullet$ \textbf{CPCB}: Central Pollution Control Board}
   \begin{table}[h]
      \centering
      \begin{tabular}{|c|c|}
        \hline
        \textbf{Abbreviation} & \textbf{Stands For} \\
        \hline
        IF & Involvement Factor \\
        \hline
        ID & Identification \\
        \hline
        CPCB & Central Pollution Control Board \\
        \hline
        mg & milligram \\
        \hline
        AC & Alternating Current \\
        \hline
        dB & Decibals \\
        \hline
        Kg & Kilograms \\
        \hline
        ABS & Acrylonitrile Butadiene Styrene \\
        \hline
        PWM & Pulse Width Modulation \\
        \hline
        IC & Integrated Circuits \\
        \hline
        PCE & perchloroethylene \\
        \hline
        TCE & Trichloroethylene \\
        \hline
        TRL & Technology Readiness Level \\
        \hline
        CAD & Computer Aided Design \\
        \hline
        NMOS & N-type Metal-Oxide-Semiconductor
 \\
        \hline
        LED & Light Emitting Diode \\
        \hline
        TCE & Tri-Chloro-Ethane \\
        \hline
        PCE & Power Conversion Efficiency \\
        \hline
      \end{tabular}
      \caption{Abbreviations}
      \label{tab:abbreviations}
    \end{table}
% \section{Index}\label{sec: indexredirect}
% \hyperref[sec:myindex]{$\bullet$ Click here to go to the Index}\hfill \pageref{sec: myindex}

% \section{Glossary}\label{sec: glossaryredirect}
% \hyperref[sec: glossary]{$\bullet$ Click here to go to the Glossary}\hfill \pageref{sec: glossary}
\clearpage


% \end{itemize}

% \lipsum[1-2]






%------------- Commandes utiles ----------------

















\newpage
\section{Mind Map}\label{sec:mindmap} 
% \begin{figure}[h]
%   \centering
%   \includegraphics[width=0.74\textwidth]{https://github.com/niteshsiingh/ELP305-Tribe-D/raw/146d5c13bdec4dc673ece1dd87fd26728a0c0d00/assets/outline_mindmap.jpg} % Replace 'example-image' with your image file name and extension
%   \caption{Outline MindMap}
%   \label{fig:outlinemindmap}
% \end{figure}
\begin{figure}[h]
    \centering
    \includegraphics[width=0.9\textwidth]{logos/P1_Outline_TribeD_MindMap.png}
    \caption{Outline Mindmap}
    \label{fig:outlinemindmap}
\end{figure}
\begin{figure}
    \centering
    \includegraphics[width=1.1\textwidth]{logos/flow2.png}
    \caption{Flowchart}
    \label{fig:outlinemindmap}
\end{figure}
\begin{figure}[htbp]
  \centering
  \begin{adjustbox}{valign=c}
    \includegraphics[width=1.1\textwidth]{logos/requirementimgg.png} % Replace 'example-image' with your image file name and extension
  \end{adjustbox}
  \caption{Requirements MindMap}
  \label{fig:reqmindmap}
\end{figure}
\begin{figure}
    \centering
    \includegraphics[width=1.1\textwidth]{logos/P1_Specifications_TribeD_MindMap.png}
    \caption{Specifications Mind Map}
    \label{fig:specsmindmap}
\end{figure}

% \begin{figure}
%     \centering
%     \href{https://github.com/niteshsiingh/ELP305-Tribe-D/raw/146d5c13bdec4dc673ece1dd87fd26728a0c0d00/assets/outline_mindmap.jpg}{\includegraphics[width=0.8\textwidth]{https://github.com/niteshsiingh/ELP305-Tribe-D/raw/146d5c13bdec4dc673ece1dd87fd26728a0c0d00/assets/outline_mindmap.jpg}}
%     \caption{Clickable Image with Link}
%     \label{fig:clickable_link}
% \end{figure}
% \begin{figure}
%     \centering
%     \includegraphics[width=0.8\textwidth]{https://github.com/niteshsiingh/ELP305-Tribe-D/raw/146d5c13bdec4dc673ece1dd87fd26728a0c0d00/assets/outline_mindmap.jpg}
%     \caption{Your Image Caption}
%     \label{fig:your_label}
% \end{figure}
% \logoentre{logos/ok3456.png}
\clearpage
\section{Project Management}\label{sec:projectmanagement}

\begin{itemize}[label=$\bullet$]
    \item {\href{https://owncloud.iitd.ac.in/nextcloud/index.php/s/aRx7A3B8AFbZ32Y}{\textcolor{blue}{Network Chart}}}
    \item {\href{https://owncloud.iitd.ac.in/nextcloud/index.php/s/pnCtc4MAoRkQte2}{\textcolor{blue}{WBS}}}
    \item {\href{https://owncloud.iitd.ac.in/nextcloud/index.php/s/MDxAqgJGXYexDLy}{\textcolor{blue}{Gantt Chart}\index{\textcolor{blue}{Gantt Chart}}}}
    \item {\href{https://owncloud.iitd.ac.in/nextcloud/index.php/s/7HQLz7ibrgMm5mY}{\textcolor{blue}{Resource Breakdown}}}
\end{itemize}


\section{Abstract}\label{sec:abstract} 
\renewcommand{\abstractname}{Cleaning Machine}

\begin{abstract}
   
This project revolves around developing a user-centric \index{\gls{user-centric}} washing machine\index{washing machine}, which involves a comprehensive analysis of the features an average user looks for. Through extensive research, we will identify key elements that resonate with the needs and preferences of the general population when searching for a washing machine\index{washing machine}.

Our initial design focuses on building a basic model, which in further iterations, can incorporate more advanced features as a result of extensive surveys and research done across the course of the project to satisfy the contemporary users' needs.
\end{abstract}

\section{Motivation}\label{sec:motivation} 
Our goal with this project is to create an advanced fabric-cleaning machine designed to wash oil stains, specifically those near the edges of manufactured cloth. This machine aims to improve the efficiency of cloth manufacturing by providing effective drying and cleaning processes while preserving the fabric. Our design’s USP is its ability to leave the areas of the cloth that are already clean untouched, which preserves the fabric’s quality and durability. Utilizing this approach helps ensure a straightforward process and reduces the resources required for drying. We currently have an Autodesk model of our machine and have researched various electrical parts and how they will be implemented practically. We have also tested various solvents in an attempt to find one which is most suitable. Finally, we intend to deliver a working model of this machine, which focuses on cleaning oil and grease stains left in cloth during the manufacturing process while focusing more on the edges of the cloth as there are higher chances of deterioration on the periphery.
\clearpage
% \section{Mechanism of the Machine}\label{sec:mechanism} 
% \clearpage


% Voici quelques commandes utiles :



% \subsection{Removal of dust using air}

%------ Pour insérer et citer une image centralisée -----
% Le premier argument est le chemin pour la photo
% Le deuxième est la hauteur de la photo
% Le troisième la légende
% Le quatrième le label
% Ici, je cite l'image \ref{fig:my_label} dans le texte.
% \begin{figure}[h!]
%     \centering
%     % \includegraphics{logos/le-mans-univ.png}
%     % \caption{Mettre une légende explicite à votre figure}
%     \label{fig:my_label}
% \end{figure}

% \subsection{Soap+Water Mechanism}
%------- Pour insérer et citer une équation --------------

% \begin{equation} \label{eq: exemple}
% \rho + \Delta = 42
% \end{equation}

% L'équation \ref{eq: exemple} est citée ici. 

% \subsection{Scrubbing}
% Les références (articles scientifiques, articles de journaux, blogs, pages web) doivent être mentionnées dans le texte par une balise \cite{maref} et fait le lien avec la citation incluse dans la bibliographie.
% \subsection{Removal of dust through air}\label{sec:removalofdust} 
% To secure the cloth in place and prevent it from being carried away by the wind, lay it flat and affix\index{\gls{affix}} it to the surface. Utilize an \gls{air blower} by directing the airflow over the cloth, with the attached blower expelling air from the top onto the fabric. For smaller pieces of fabric, a 500W mini blower\index{blower}, priced at Rs 500, is an effective solution. Alternatively, a manual\index{manual} approach involves installing a high-speed fan within a pipe for a similar effect.


% \subsection{Soap + Water mixture}\label{sec:soapwater} 
% % \logoentre{logos/opop1212.png}
% \begin{figure}
%     \centering
%     \begin{subfigure}{0.4\textwidth}\label{sec:figone}
%         \centering
%         \includegraphics[width=\linewidth]{logos/sprinkler fig1.png}
%         \caption{Isometric View\index{\gls{Isometric View}1}}
%     \end{subfigure}\hspace{0.1\textwidth}%
%     \begin{subfigure}{0.5\textwidth}\label{sec:figtwo}
%         \centering
%         \includegraphics[width=\linewidth]{logos/sprinkler fig2.png}
%         \caption{Isometric View\index{\gls{Isometric View}}2}
%     \end{subfigure}
%     \caption{Sprinkler}
% \end{figure}
% The fundamental concept behind this method is to ensure comprehensive\index{comprehensive} cleaning by spreading the soap solution evenly on both sides of the fabric. To execute this, a soap solution\index{soap solution} is meticulously prepared above the targeted cloth. This solution with a predetermined ratio of soap to water facilitates effective cleansing. Employing a specialized nozzle mechanism\index{\gls{nozzle mechanism}}, the soap solution\index{soap solution} is methodically sprinkled onto the fabric evenly in both directions making it more effective in removing dirt, stains\index{stains}. The end result comes out to be a thorough and uniform cleaning mechanism.

% \subsection{Scrubbing of the cloth}\label{sec:scrubbing} 
% The dimensions of the platform will be such that its length matches its width. A circular sponge-type scrubber\index{scrubber} mounted on a rod above the platform will be employed to initiate the cleaning process\index{process}. We will use stepper\index{stepper} motors\index{motors} to enable the scrubber to move perpendicular to the cloth's motion. The cloth will be positioned beneath the scrubber, and its movement will be halted. It is important to note that the scrubber\index{scrubber} is designed to be smaller than the cloth. The scrubber\index{scrubber} will then traverse in the perpendicular direction, covering a strip of the cloth. After completing this operation, it will advance by a predetermined distance, and this sequential process will continue until the entire cloth has been thoroughly scrubbed using the motor\index{motor}. 

% \subsection{Water Mechanism}\label{sec:water} 
% The scrubbed fabric retaining traces\index{\gls{traces}} of soap, undergoes exposure to high-pressure water from a nozzle\index{\gls{nozzle}}. Subsequently, the cloth is guided through a wiper to eliminate any surplus moisture and soap solution\index{soap solution}. The combined unit, comprising both the nozzle\index{\gls{nozzle}} and wiper, moves back and forth across the fabric for several iterations, with the exact number determined during the \gls{testing phase}.
% \subsection{Drying}\label{sec:drying} 
% The device produces warm air directed towards damp surfaces using a hot air-drying\index{\gls{hot air-drying}} method. The same mechanism can be understood\index{understood} as the one used in a hair dryer.  This targeted application of heat speeds up the evaporation\index{evaporation} process of water molecules. The elevated\index{elevated} temperature boosts the energy of the water, facilitating its swift transition from liquid to vapor. This mechanism effectively eliminates moisture, making it a fast and efficient technique for drying fabrics. 

% \clearpage
\section{Requirements for the idea}\label{sec:requirements} 
    

\subsection{ Input Specifications}\label{sec:inputspec}
\subsubsection{Material Specifications:}\label{sec:matspec}
    \begin{itemize}[label=$\bullet$]
      \item Newly manufactured white unbleached cotton with single-ply, Denier\index{\gls{Denier}} 60, and a thread count\index{\gls{threadcount}} 400.
    \end{itemize}
\subsubsection{Stains}\label{sec:matspec}
    \begin{itemize}[label=$\bullet$]
      \item Only oil and grease stains present on the edges of the cloth need to be removed

    \end{itemize}
  
  \subsubsection{Dimensions:}\label{sec:dimensions}
    \begin{itemize}[label=$\bullet$]
      \item Cloth is either available on rollers(2m*10m) or it can be assumed as an infinite sheet supply of width 2 m.

    \end{itemize}
  
  \subsubsection{ Cloth Characteristics:}\label{sec:clothchar}
    \begin{itemize}[label=$\bullet$]
      \item Free from foul odour, slightly damp, and without buttons, zippers, or attachments.
    \end{itemize}
  
  \subsubsection{ Cleaning Limitations:}\label{sec:cleanlim}
    \begin{itemize}[label=$\bullet$]
      \item Maximum weight for cleaning is set at 11 kg dry, with stains\index{stains} limited to those occurring during manufacturing\index{manufacturing}.
    \end{itemize}

% \subsection{ Client Preferences}\label{sec:clientpref}
%     \subsubsection{}{Cost and Service Preferences:}\label{sec:cnspref}
%     \begin{itemize}
%       \item Preference for the washing machine to be offered at zero cost, requiring no servicing time and no maintenance.
%       \item Actual prices are expected to depend on the provider, with alternatives considered if costs are excessively high.
%     \end{itemize}

\subsection{ Outputs Requirement}\label{sec:outputreq}
\subsubsection{ Desired Output:}\label{sec:desout}
    \begin{itemize}[label=$\bullet$]
      \item A cleaned and dry cloth wound on rollers.
    \end{itemize}
    
% \newpage
\subsubsection{Client Responsibilities:}\label{sec:clientresp}
    \begin{itemize}[label=$\bullet$]
      \item Treating discharged greywater \index{\gls{graywater}}, managing lint, and ensuring the returned cloth is wrinkle-free and bone-dry\index{\Gls{bone-dry}}.
    \end{itemize}

\subsection{ Power Requirements}\label{sec:powerreq}
\subsubsection{Voltage and Phase Requirements:}\label{sec: vnpreq}
    \begin{itemize}[label=$\bullet$]
      \item The washing machine should operate on 220VAC 15A, with the option for 440VAC 3-phase available at an additional cost.
    \end{itemize}
  
  \subsubsection{Operational Expectations:}\label{sec: opexp}
    \begin{itemize}[label=$\bullet$]
      \item They are expected to run continuously, 24/7, with an emergency shutdown initiated using a 1-button process.
    \end{itemize}

\subsection{Logistical Requirements}\label{sec:logireq}
\subsubsection{Machine Type and Features:}\label{sec:machtype}
    \begin{itemize}[label=$\bullet$]
      \item I prefer an automatic washing machine with minimal water usage and no need for portability or a programmable timer.
    \end{itemize}
\subsubsection{Washing Medium Features:}\label{sec:wmtype}
    \begin{itemize}[label=$\bullet$]
      \item There are no restrictions on the washing medium, but costs may be incurred for using rare solvents, focusing on overall cost-effectiveness.
    \end{itemize}

\subsection{ Environmental Requirements}\label{subsec:envireq}
\subsubsection{ Noise and Compliance:}\label{sec:noisecomp}
    \begin{itemize}[label=$\bullet$]
      \item Noise levels should not exceed 75 dB.
      \item Must comply with local regulations, including those set by the Central Pollution Control Board (CPCB).
    \end{itemize}
  
  \subsubsection{Sustainability Preferences:}\label{sec:suspref}
    \begin{itemize}[label=$\bullet$]
      \item Preference for cold water washing, sustainable components, and optimization of energy consumption, robustness, and durability.
    \end{itemize}

\subsection{Site Requirements}\label{sec:sitereq}
\subsubsection{Essentials for the Site:}\label{sec:essenforsite}
    \begin{itemize}[label=$\bullet$]
      \item Adequate power supply, suitable drainage, and specific design parameters.
    \end{itemize}
  
 \subsubsection{Water Source and Structural Considerations:}\label{sec:watersource}
    \begin{itemize}[label=$\bullet$]
      \item The water source was specified as having 60 mg CaCO3/l hardness, with an overhead tank and a 50,000-liter refillable capacity at 35 meters.
      \item Structural considerations include material selection and the ability to withstand the maximum cloth weight.
    \end{itemize}

\subsection{Time Requirements}\label{sec:timereq}
\subsubsection{Cleaning and Setup Times:}\label{sec: cnstimes}
    \begin{itemize}[label=$\bullet$]
      \item Cleaning and drying time set at 45 minutes.
      \item Setup time should be at most 1 day.
    \end{itemize}
\subsubsection{Design Time Requirements}\label{sec: destimereq}
    \begin{itemize}[label=$\bullet$]
      \item Cleaning and Drying time: Atmost 45 minutes.
      \item Usage time: 24 hrs a day
      \item Setup Time: As little time as possible, no more than 1 day.
    \end{itemize}
  
  \subsubsection{Time to Market Requirements}\label{sec: timemarketreq}
    % \begin{itemize}
    %   \item They are expected to run continuously, 24/7, with an emergency shutdown initiated using a 1-button process.
    % \end{itemize}
  \subsubsection{LifeTime Requirements}\label{sec: lifetimereq}
    \begin{itemize}[label=$\bullet$]
      \item Expected Lifetime: The machine is expected to last atlest 6 years.
      \item Service Hours and Cost: No more than 6 hours per year and there isn’t an explicit cost constraint for the servicing. years.
    \end{itemize}
  \subsubsection{ End of Life Requirements}\label{sec: endoflifereq}
    \begin{itemize}[label=$\bullet$]
      \item Replacement for Old Machine: Client could be interested in replacing the old machine for a new one at a discounted price.
      \item Parts' Availability:Parts of the machine should be available for 10 years to enable servicing.
    \end{itemize}
    

% \subsection{Lifetime Requirements}\label{sec:lifereq}
% \subsubsection{Lifespan and Service Limits:}\label{sec:lifespanandservice}
%     \begin{itemize}
%       \item The machine's lifespan should be at least 6 years.
%       \item At most, 6 hours of service per year.
%     \end{itemize}

\subsection{Other Non-Functional Requirements}\label{sec:otherreq}
\subsubsection{Miscellaneous Considerations:}\label{sec:misccons}
    \begin{itemize}[label=$\bullet$]
      \item Dimensions and the inclusion of a stand or wheels are left to the designer's discretion.
      \item We need to clean only edges of the cloth rather the whole cloth
      \item only oil stains and grease stains are need to be removed
      \item Cloth is either available on rollers(2m*10m) or it can be assumed as an infinite sheet supply of width 2 m.

    \end{itemize}
\clearpage
\section{Overview}
We propose the development of an innovative cloth cleaning machine that can be used to clean oil stains (which occur near the edges) off of manufactured cloth after the manufacturing process. Our design consists of rollers, driving motors, a guiding frame \index{\gls{guiding frame}} (used to fold the cloth in half vertically), wiping and cleaning surface (brushes/sponge), a soap water mixture compartment, a water compartment, and a drying chamber. This automated system aims to streamline the cloth manufacturing process, ensuring efficient cleaning and drying during the manufacturing process while minimising the impact on the fabric. The USP of our design is that the regions of the cloth that are guaranteed to come in clean are untouched in the process, which preserves the quality and durability of the cloth. This approach also ensures that drying requires less effort and resources in comparison to other approaches.
\subsection{Key Components}
\begin{enumerate}[label=\arabic*.]

    \item \textbf{Rollers with Motor Control:}
    \begin{itemize}[label=$\bullet$]
        \item Two rollers placed on either side of the machine.
        \item Motor-controlled to regulate the speed of the cloth movement.
    \end{itemize}

    \item \textbf{Cloth Attachment:}
    \begin{itemize}[label=$\bullet$]
        \item Cloth is securely attached to the rollers upon loading in batches, ensuring uniform tension.
    \end{itemize}

    \item \textbf{Guiding Frame:}
    \begin{itemize}[label=$\bullet$]
        \item Positioned between the rollers upon viewing from the side and placed parallelly between the two vertical walls when viewed from the top.
        \item The frame when viewed from the side looks like a smoothened plateau with a long flat top and curved sides (coming from and going towards the input and output rollers respectively).
        \item Upon viewing from the top, it looks like the edge of a railing.
        \item The frame is slightly angled and is broader at the bottom than at the top.
        \item All edges on the frame are filleted \index{\gls{filleted}} and smoothened to ensure that the cloth doesn’t rip or tear or get stuck while it slides over the frame due to the pull of the motor.
        \item It is shaped this way to guide the fabric to smoothly rise in height from the horizontal roller configuration at the input to the folded vertical configuration in the cleaning and drying chamber.
        \item This vertical folding ensures that the stained edges are hanging on the two sides of the frame symmetrically with the entire cloth folded along the midline and suspended vertically by the normal reaction from the sleek frame.
        \item The part of the fabric towards the centre in the horizontal configuration, now slides at the top of the smooth frame as the rollers on the other end pull it at a constant speed and is unaffected by the cleaning and drying process.
        \item Once the cloth crosses the drying chamber, the guiding frame is shaped in such a way that it lowers the cloth from the raised frame back to its horizontal configuration onto the rollers.
    \end{itemize}

    \item \textbf{Cleaning Mechanism:}
    \begin{itemize}[label=$\bullet$]
        \item Brushes along the edges of the frame at a fixed distance from the height (coincides with the stained edge of the cloth).
        \item The parallel wall also has brushes at the same vertical height, and the two brushes hold on to the edge of the cloth as it moves under the influence of the rollers and scrubs against the brushes, hence getting cleaned.
        \item Solvent drizzled from the top onto the stained part of the cloth). The solvent mixture and water are dispensed in succession and recursively creating stages along the length of the frame. (first “x” cm-> soap, next “y” cm-> water, next “x” cm-> soap, etc. )
        \item Brushes act as scrubbers to enhance cleaning effectiveness.
    \end{itemize}

    \item \textbf{Cleaning Fluid Compartment:}
    \begin{itemize}[label=$\bullet$]
        \item Alternating compartments for soap water mixture and water.
        \item The soap compartments have brushes along the wall, while the water compartment consists of high-pressure water nozzles to take out the soap from the cloth.
        \item Ensures proper cleaning of the cloth during the process.
    \end{itemize}

    \item \textbf{Drying Chamber:}
    \begin{itemize}[label=$\bullet$]
        \item Located after the cleaning mechanism.
        \item Equipped with air blowers \index{\gls{air blowers}} to blow hot air onto the cloth.
        \item Ensures quick and effective drying.
    \end{itemize}

    \item \textbf{Guiding Fixtures \index{\gls{Fixtures}}}
    \begin{itemize}[label=$\bullet$]
        \item Transition the cloth from the input roller to the cleaning chamber of the frame and from the drying chamber of the frame to the rollers.
        \item Facilitates a smooth movement of the cloth.
    \end{itemize}

    \item \textbf{Waste Tub:}
    \begin{itemize}[label=$\bullet$]
        \item A waste tub at the bottom to collect residual drippings and lint.
    \end{itemize}

\end{enumerate}
\begin{figure}[h]
    \centering
    \includegraphics[width=0.6\textwidth]{logos/top view.png}
    \caption{Top View}
    \label{fig:outlinemindmap}
\end{figure}
\begin{figure}[h]
    \centering
    \includegraphics[width=0.6\textwidth]{logos/side view.png}
    \caption{side view}
    \label{fig:outlinemindmap}
\end{figure}
\begin{figure}[h]
    \centering
    \includegraphics[width=0.6\textwidth]{logos/isometric 1.png}
    \caption{Isometric View}
    \label{fig:outlinemindmap}
\end{figure}
\clearpage
\subsection{WorkFlow}
\begin{enumerate}[label=$\bullet$]

    \item Cloth is loaded onto input rollers over the guiding frame, through the guiding fixtures, and onto the output roller. (In order to prevent wastage of a significant initial length of the cloth roll, we can attach it to a clean dummy cloth as long as the machine with the help of a speed punching system and later detach the dummy cloth and reuse it for the next batch—This is not relevant to the demonstration but very relevant during industrial scaling.)

    \item Then a lever/switch is activated which brings the two walls closer to the guiding frame with the cloth attached to it and the horizontal line of brushes lock onto each hanging edge of the cloth securing it in place.

    \item Motors are powered on, and the cloth starts sliding on the frame while being kept in place by the walls and guiding fixtures.

    \item Solvent is drizzled from the top with targeted precision on the edges, and brushes act as scrubbers for thorough cleaning as the cloth slides over them.

    \item The cloth passes through multiple alternating soap water mixture and water compartments and gets recursively cleaned for better results.

    \item The cloth then enters the drying chamber, where hot air is blown to expedite the drying process.

    \item After drying, the cleaned and dried cloth moves through guiding fixtures onto the rollers for subsequent manufacturing processes.

\end{enumerate}
\subsection{Benefits}
\begin{enumerate}[label=$\bullet$]
    \item Improved cloth cleaning efficiency.
    \item Minimized impact on the fabric during the cleaning process.
    \item Streamlined manufacturing workflow.
    \item Enhanced drying capabilities for increased production speed.
\end{enumerate}
\textbf{Calculation of order of magnitude of the length of chamber:}
\begin{itemize}[label=$\bullet$]
    \item Load in one batch = 11 kg
    \item Breadth of cloth roll = 2 m
    \item Areal density \index{\gls{Areal density}} of Single ply cotton cloth unbleached, denier 60, thread count 400 = 0.180 kg/m\textsuperscript{2} 
    \item Time provided for cleaning = 45 minutes
    \item \( l = \frac{90 \times 11}{0.18 \times 2 \times 45 \times 60} \) m
\end{itemize}
\clearpage
\section{Component Analysis}\label{sec:compana}
\subsection{Roller}\label{sec:roller}
\begin{enumerate}
  \item The roller will roll the washed cloth, coming through the conveyor belt.
  \item A controlled DC motor\index{motor} will be used to drive the roller.
  \item Appropriately select the dimensions of the roller, like the diameter of the roller and its length, based on the conveyor width.
  \item Choose a proper outer covering for the roller, which can provide a better grip and friction for the cloth.
  \item The cloth will also be straightened using 1 or 2 uncontrolled rolling cylinders which can provide the requisite tension in the fabric and guide the fabric onto the roller.
\end{enumerate}
\begin{figure}[h]
    \centering
    \includegraphics[width=0.6\textwidth]{logos/roller1img.png}
    \caption{ ( A textile rewinding machine )                    
}
    \label{fig:outlinemindmap}
\end{figure}
\subsubsection{Controlling DC Motor\index{Motor} using Arduino\index{Arduino}}\label{sec:contdcmotor}

To control the speed of a DC motor\index{motor} using Arduino\index{\gls{Arduino}}, we need to adjust the input voltage\index{voltage} supplied to the motor\index{motor}. 
We can control the input voltage\index{voltage} with a pulse-width modulated (PWM) signal.
To change the speed of the DC motor\index{motor} we need to change the amplitude of the input voltage\index{voltage} that is applied to the motor\index{motor}.
A common technique to do that is PWM (Pulse Width Modulation). In PWM the applied voltage\index{voltage} is adjusted by sending a series of pulses so the output voltage is proportional pulse width generated by the microcontroller that is also known as duty cycle.
\begin{figure}[h]
    \centering
    \includegraphics[width=0.8\textwidth]{logos/dutyimg.png}
    \caption{(A square wave with 50\% duty cycle)
}
    \label{fig:outlinemindmap}
\end{figure}
The higher the duty cycle, the higher the average voltage applied to the DC motor\index{motor} (resulting in higher speed) and the shorter the duty cycle, the lower the average voltage applied to the DC motor\index{motor} (resulting in lower speed).
\subsubsection{Arduino and L293D Circuit Diagram}\label{sec:adriono}
A common and cheap solution to drive motor\index{motor}s and efficiently control them, is to use a Motor\index{Motor} Controller \index{\gls{Controller}} module along with Arduino. L293D Motor\index{Motor} driver module is a readily available IC which can be easily interfaced with Arduino, to control the various aspects of DC motors\index{motors} like speed, direction and braking. It is designed to provide bidirectional\index{bidirectional} drive currents of up to 600-mA at voltages from 4.5 V to 36V.
Below is an example of a circuit diagram to drive multiple motors\index{motors} from a single module, and Arduino\index{Arduino} code to interface a motor\index{motor} with the module.
\begin{figure}[h]
    \centering
    \includegraphics[width=0.5\textwidth]{logos/adriono.jpg}
    \caption{( Sample circuit diagram of motor control)}
    \label{fig:outlinemindmap}
\end{figure}
\item \href{https://github.com/naunidhsingh03/ELP305-TribeD-Resources/blob/5ba1988fe283faba21ba7098978bb225e509d5cb/Codes/final_roller.ino}{\textcolor{blue}Click here for arduino code}}
% \begin{lstlisting}
% int enA = 9;
% int in1 = 8;
% int in2 = 7;
% int enB = 3;
% int in3 = 5;
% int in4 = 4;
% void setup() {
%   // Set all the motor control pins to outputs
%   pinMode(enA, OUTPUT);
%   pinMode(enB, OUTPUT);
%   pinMode(in1, OUTPUT);
%   pinMode(in2, OUTPUT);
%   pinMode(in3, OUTPUT);
%   pinMode(in4, OUTPUT);
%   // Turn off motors - Initial state
%   digitalWrite(in1, LOW);
%   digitalWrite(in2, LOW);
%   digitalWrite(in3, LOW);
%   digitalWrite(in4, LOW);
% }

% void loop() {
%   directionControl();
%   delay(1000);
%   speedControl();
%   delay(1000);
% }

% // This function lets you control the spinning direction of motors
% void directionControl() {
%   // Set motors to maximum speed
%   // For PWM maximum possible values are 0 to 255
%   analogWrite(enA, 255);
%   analogWrite(enB, 255);
%   // Turn on motor A & B
%   digitalWrite(in1, HIGH);
%   digitalWrite(in2, LOW);
%   digitalWrite(in3, HIGH);
%   digitalWrite(in4, LOW);
%   delay(2000);
%   // Now change motor directions
%   digitalWrite(in1, LOW);
%   digitalWrite(in2, HIGH);
%   digitalWrite(in3, LOW);
%   digitalWrite(in4, HIGH);
%   delay(2000);
%   // Turn off motors
%   digitalWrite(in1, LOW);
%   digitalWrite(in2, LOW);
%   digitalWrite(in3, LOW);
%   digitalWrite(in4, LOW);
% }

% // This function lets you control the speed of the motors
% void speedControl() {
%   // Turn on motors
%   digitalWrite(in1, LOW);
%   digitalWrite(in2, HIGH);
%   digitalWrite(in3, LOW);
%   digitalWrite(in4, HIGH);
%   // Accelerate from zero to maximum speed
%   for (int i = 0; i < 256; i++) {
%     analogWrite(enA, i);
%     analogWrite(enB, i);
%     delay(20);
%   }
%   // Decelerate from maximum speed to zero
%   for (int i = 255; i >= 0; --i) {
%     analogWrite(enA, i);
%     analogWrite(enB, i);
%     delay(20);
%   }
%   // Now turn off motors
%   digitalWrite(in1, LOW);
%   digitalWrite(in2, LOW);
%   digitalWrite(in3, LOW);
%   digitalWrite(in4, LOW);
% }
% \end{lstlisting}
% \textbf{$\bullet$ Electrical Specifications for the actual machine:}
% \begin{itemize}
%   \item Rolling capacity: \textasciitilde60-80 meters/min
%   \item Rewind Diameter: 1.8-2 meters
%   \item Power consumption: 220V
%   \item Drive power: 0.5-1 hp
%   \item Cost: A standalone, dedicated textile rewinding machine may cost around Rs 1 lakh in India, but since it would be integrated with the other components of our machine, the cost would reduce significantly.
% \end{itemize}
% \textbf{$\bullet$ Electrical specifications for the prototype:}
% \begin{itemize}
%   \item Operating .\index{voltage}: max 36V (this is the value for which the module is configured)
%   \item Current rating: max 600mA per channel
%   \item Motor: If a small DC motor is used, a no-load speed of \textasciitilde9000 rpm, and a loaded speed of \textasciitilde5000 rpm can be achieved.
%   \item Cost: Since the driver module can be used simultaneously for other motors in the system, the average cost of the roller would be \textasciitilde Rs. 200.
% \end{itemize}
\subsection{Soap Water Requirements}
\subsubsection{A. For splitted cloth pieces using sensor}
\begin{figure}[h]
    \centering
    \includegraphics[width=0.87\textwidth]{logos/soap1.png}
    \caption{  For splitted cloth pieces using sensor}
    \label{fig:outlinemindmap}
\end{figure}
\href{https://github.com/naunidhsingh03/ELP305-TribeD-Resources/blob/5ba1988fe283faba21ba7098978bb225e509d5cb/Codes/solvent_dispenser_old_design_1.ino}{\textcolor{blue}{Click here for the code reference}}
\subsubsection*{Explanation:}
\begin{itemize}[label=$\bullet$]
    
\item When the distance is less than 10cm, we have to turn on the MOSFET, and otherwise, we have to turn off the MOSFET. We will also use the on-board LED connected to pin 13 and toggle it along with the MOSFET so that we can ensure if the MOSFET is in the turned-on or turned-off state.

\item Inside the main loop function, we call for the function called \texttt{measure\_distance()}. This function uses the US sensor to measure the distance of the object in front of it and updates the value to the variable \texttt{distance}.

\item The input or the detection will send a sonic blast of Ultrasonic signals into the air, which will get reflected by the object in front of it, and the echo pin will pick up the signals reflected by it. Then we use the time taken value to calculate the distance of the object ahead of the sensor.

\item Once the distance is calculated, we have to compare the value of distance using a simple if statement. If the value is less than 10cm, we make the MOSFET and LED go high. In the following else statement, we make the MOSFET and LED go low.
\end{itemize}
\clearpage
\subsubsection{B. Without Sensor, using push button : }
\begin{figure}[h]
    \centering
    \includegraphics[width=0.87\textwidth]{logos/soap2.png}
    \caption{ circuit for Using push button}
    \label{fig:outlinemindmap}
\end{figure}
\href{https://github.com/naunidhsingh03/ELP305-TribeD-Resources/blob/5ba1988fe283faba21ba7098978bb225e509d5cb/Codes/solvent_spinkler_new_design_1.ino}{\textcolor{blue}{Click here for code reference}}
\subsubsection*{Explanation}
\begin{itemize}[label=$\bullet$]
    \item A push button is employed to toggle the relay, managing the on/off state of the pump.
    \item The relay is cycled at regular intervals, simulating control over a device, such as a solvent pump.
     To activate the relay, the relay pin (control pin) is set to HIGH.
     To deactivate the relay, the relay pin is set to LOW.
\end{itemize}

\begin{figure}[h]
    \centering
    \begin{subfigure}{0.4\textwidth}\label{sec:figone}
        \centering
        \includegraphics[width=\linewidth]{logos/sprinkler fig1.png}
        \caption{Isometric View\index{\gls{Isometric View}1}}
    \end{subfigure}\hspace{0.1\textwidth}%
    \begin{subfigure}{0.5\textwidth}\label{sec:figtwo}
        \centering
        \includegraphics[width=\linewidth]{logos/sprinkler fig2.png}
        \caption{Isometric View\index{\gls{Isometric View}}2}
    \end{subfigure}
    \caption{Sprinkler}
\end{figure}
\clearpage
\subsection{Scrubber}
\begin{itemize}[label=$\bullet$]
    \item Static stationary brush scrubber is an industrial cleaning tool designed for efficient cloth cleaning. This device features stationary brushes that remain fixed during operation, providing a stable cleaning surface. The static design allows for controlled and targeted cleaning of fabrics, ensuring uniform and effective removal of dirt and contaminants. This type of scrubber is commonly employed in industrial settings where precision and consistency in cloth cleaning are essential for maintaining high standards of hygiene and product quality.
    \item Stattonary brushes would be used near the edges of the cloth.
    \item Static cleaning mechanism
    \item Zero power requirement
    % \item Specifications:
    
\end{itemize}

% \begin{figure}[h]
%     \centering
%     \includegraphics[width=0.6\textwidth]{logos/scrubber.png}
%     \caption{Tunnel Dryer}
%     \label{fig:outlinemindmap}
% \end{figure}
\subsection{Blower}\label{sec:blower}
\begin{itemize}[label=$\bullet$]
  \item We will be controlling the speed of the DC motor using a potentiometer for the purpose of a blower in our washing machine to remove heavy dust particles from the cloth piece.
  \item Potentiometer has three terminals. Outer two terminals are for power supply, and the middle terminal is the output.
  \item As we rotate the knob of the potentiometer, the resistance between the middle terminal and one of the outer terminals changes. This change in resistance controls the voltage supplied to the motor, which in turn controls its speed.
  \item We can connect multiple DC motors in our system by making slight modifications in the wiring.
  \item A Transistor is used for more efficient control of the motor speed. By controlling the current flow to the motor, transistors can prevent overloading and overheating, enhancing the motor’s lifespan \index{\gls{lifespan}}.
\end{itemize}
\clearpage

\subsubsection{Tinkercad:}
\begin{figure}[h]
    \centering
    \includegraphics[width=0.76\textwidth]{logos/tinkercad model.png}
    \caption{Tinkercad model for blower based on DC motor speed control using potentiometer
}
    \label{fig:outlinemindmap}
\end{figure}
\hspace{1cm}\href{https://github.com/naunidhsingh03/ELP305-TribeD-Resources/blob/5ba1988fe283faba21ba7098978bb225e509d5cb/Codes/blower.ino}{\textcolor{blue}{Click here for code references}}
% \begin{lstlisting}
% const int poten = A3;
% int var;

% void setup()
% {
%   pinMode(6, OUTPUT);
% }

% void loop()
% {
%   var = analogRead(poten);
%   analogWrite(6, var);
% }
% \
% \end{lstlisting}
\clearpage
\subsection{Dryer}
We are planning to use the configuration of tunnel dryer to dry the clothes. The power and torque requirements of the motor used in blower and power requirements of the heater will depend on the time needed to dry the cloth, rate at which the cloth is being fed, width and height of the chamber, final moisture content and initial moisture content.
Also, since counter current configuration is most efficient, we will be using the same in our design. Using tunnel dryers also allows us to move the conveyor belt slowly as it is very efficient in processing materials taking long drying time and thus requiring lesser motor drive.
\begin{figure}[h]
    \centering
    \includegraphics[width=0.8\textwidth]{logos/dryer.png}
    \caption{Tunnel Dryer}
    \label{fig:outlinemindmap}
\end{figure}
Optimisation for power requirements will be done once design specs are provided and it would be based on the mathematical modelling and simulations done to observe the humidity content with rate of air flow and power input to heater and blower.
To control the heater we will use Arduino, a temperature sensor (thermocouple), a relay module, battery and bunch of connecting wires.
One of the circuits which can be used is as follows:
One can have LCD display to keep track of any errors in the functioning.
Arduino code for controlling heater is as follows:

\href{https://github.com/naunidhsingh03/ELP305-TribeD-Resources/blob/5ba1988fe283faba21ba7098978bb225e509d5cb/Codes/dryer.ino}{\textcolor{blue}{Click here for code reference}}
\clearpage
\begin{figure}[h]
    \centering
    \includegraphics[width=0.8\textwidth]{logos/dryerimgg.png}
    \caption{Control circuit for heater
}
    \label{fig:outlinemindmap}
\end{figure}
For running and controlling the blower we need to have an Arduino controlling motor, anemometer, battery and bunch of wires.

\begin{figure}[h]
    \centering
    \includegraphics[width=0.8\textwidth]{logos/dryer2img.png}
    \caption{Control Circuit for a DC motor/fan.}
    \label{fig:outlinemindmap}
\end{figure}
\hspace{1cm}\href{https://github.com/naunidhsingh03/ELP305-TribeD-Resources/blob/b90ebc4b82fa8bee4717be5c5bfab0fe1fc3e671/Codes/dryer_code_2.ino}{\textcolor{blue}{Click here for Code references}}

\clearpage
\section{Solvent Research}
Some of the cleansing agents that were researched and tested for oil and grease stains

\begin{enumerate}
    \item \textbf{Lipase \index{\gls{Lipase}} Enzyme}
    \begin{itemize}[label=$\bullet$]
        \item \textbf{Pros:} Lipase is an enzyme that breaks down oil, making it a potential solution for cleaning oil and grease stains.
        \item \textbf{Cons:} Requires slightly warm water (\textasciitilde40-50\textdegree C) for optimal enzyme action, which may be challenging to maintain.
    \end{itemize}

    \item \textbf{Spot Remover (e.g., Shout) and Hot Water}
    \begin{itemize}[label=$\bullet$]
        \item \textbf{Pros:} Effective in removing both oil and grease stains. Hot water and brushing enhance results.
        \item \textbf{Cons:} Increases energy requirements and costs due to hot water usage. Spot removers add to the overall cost.
    \end{itemize}

    \item \textbf{Spot Lifters (e.g., K2R Spot Lifter) followed by Dry Brushing}
    \begin{itemize}[label=$\bullet$]
        \item \textbf{Pros:} Overcomes the need for scrubbing in hot water. After application, a short waiting period lifts the stain, requiring only brushing in the end.
        \item \textbf{Cons:} Costly \texttt{(\textasciitilde{3300} Rs for 150 ml)} and availability may be an issue for large quantities due to US manufacturing.
    \end{itemize}

    \item \textbf{Baby Powder/Cornstarch/Salt/Vinegar followed by Washing}
    \begin{itemize}[label=$\bullet$]
        \item \textbf{Pros:} Effective in cleaning oil and grease stains significantly.
        \item \textbf{Cons:} Time-consuming, requiring soaking and multiple iterations to remove stains.
    \end{itemize}

    \item \textbf{WD-40 or Lighter Fluid}
    \begin{itemize}[label=$\bullet$]
        \item \textbf{Pros:} Effective in cleaning oil and grease stains significantly.
        \item \textbf{Cons:} Requires 20 minutes of soaking time, and hot water is needed for washing.
    \end{itemize}

    \item \textbf{PCE (Perchloroethylene)}
    \begin{itemize}[label=$\bullet$]
        \item \textbf{Pros:} Nonflammable liquid solvent widely used in dry cleaning, effective in removing oil and grease stains in small quantities.
        \item \textbf{Cons:} Extended exposure to large quantities may cause irritation to eyes, skin, throat, nose, and respiratory system.
    \end{itemize}

    \item \textbf{Handwash (Dettol)}
    \begin{itemize}[label=$\bullet$]
        \item \textbf{Pros:} Biodegradable, environmentally friendly. Can be washed and reused multiple times. Soft and less abrasive.
        \item \textbf{Cons:} May be more prone to staining depending on material and color.
    \end{itemize}

    \item \textbf{Dishwasher (Vim Drop)}
    \begin{itemize}[label=$\bullet$]
        \item \textbf{Pros:} Effective in removing both oil and grease stains by scrubbing with Vim Drop and water with normal pressure.
    \end{itemize}

    \item \textbf{Trichloroethylene (TCE)}
    \begin{itemize}[label=$\bullet$]
        \item \textbf{Pros:} Stain remover and degreaser. Evaporates quickly, leaving behind a dry surface.
        \item \textbf{Cons:} Linked to various health risks, including respiratory, neurological, and reproductive effects. Considered a potential human carcinogen \index{\gls{carcinogen}}. Flammable and persistent in the environment.
    \end{itemize}

    \item \textbf{Isopropyl (Rubbing Alcohol 68-72\%)}
    \begin{itemize}[label=$\bullet$]
        \item \textbf{Pros:} Natural degreasing agent.
        \item \textbf{Cons:} Ineffective in practical experiments on grease and oil-stained cotton cloth. May be effective after longer soaking, making it inefficient due to time constraints.
    \end{itemize}
    \item \textbf{Detergent (Surf Excel)}
    \begin{itemize}[label=$\bullet$]
        \item \textbf{Pros:} Effective on oil stains with slight scrubbing.
        \item \textbf{Cons:} The idea was rejected as it was not effective on grease stains, even after hard scrubbing.
    \end{itemize}
    \item \textbf{Acetone}
    \begin{itemize}[label=$\bullet$]
        \item \textbf{Pros:} As it is volatile, it would be quite convenient in the drying stage.

        \item \textbf{Cons:} We have experimentally seen that it does not work efficiently on grease stains

    \end{itemize}
\end{enumerate}
\subsection{We tested many solvents and the results are as follows:}
\
% \clearpage
\begin{itemize}[label=$\bullet$]
        \item \textbf{WD-40} 
    \end{itemize}
\begin{figure}[h]
    \centering
    \begin{subfigure}{0.265\textwidth}
        \centering
        \includegraphics[width=\linewidth]{logos/wd40before.jpg}
        \caption{Before}
    \end{subfigure}\hspace{0.1\textwidth}%
    \begin{subfigure}{0.265\textwidth}
        \centering
        \includegraphics[width=\linewidth]{logos/wd40after.jpg}
        \caption{After}
    \end{subfigure}
    \caption{Application of WD-40}
\end{figure}
\clearpage
% \subsection{We tested many solvents and the results are as follows:}
\begin{itemize}[label=$\bullet$]
        \item \textbf{Acetone} 
    \end{itemize}
\begin{figure}[h]
    \centering
    \begin{subfigure}{0.2\textwidth}
        \centering
        \includegraphics[width=\linewidth]{logos/acetonebefore.jpg}
        \caption{Before}
    \end{subfigure}\hspace{0.1\textwidth}%
    \begin{subfigure}{0.2\textwidth}
        \centering
        \includegraphics[width=\linewidth]{logos/acetone after.jpg}
        \caption{After}
    \end{subfigure}
    \caption{Application of acetone}
\end{figure}
% \clearpage
\begin{itemize}[label=$\bullet$]
        \item \textbf{Dishwash(Vim)} 
    \end{itemize}
\begin{figure}[h]
    \centering
    \begin{subfigure}{0.2\textwidth}
        \centering
        \includegraphics[width=\linewidth]{logos/vimbefore.jpg}
        \caption{Before}
    \end{subfigure}\hspace{0.1\textwidth}%
    \begin{subfigure}{0.2\textwidth}
        \centering
        \includegraphics[width=\linewidth]{logos/vimafter.jpg}
        \caption{After}
    \end{subfigure}
    \caption{Application of dishwasher(vim)}
\end{figure}
% \clearpage
\begin{itemize}[label=$\bullet$]
        \item \textbf{Some more applications} 
    \end{itemize}

\begin{figure}
    \centering
    \begin{subfigure}{0.1\textwidth}
        \centering
        \includegraphics[width=\linewidth]{logos/greaseisopropyl.jpg}
        \caption{ after applying isopropyl}
    \end{subfigure}\hspace{0.1\textwidth}%
    \begin{subfigure}{0.1\textwidth}
        \centering
        \includegraphics[width=\linewidth]{logos/greasedet.jpg}
        \caption{after using detergent }
    \end{subfigure}
    \caption{Image of grease}
\end{figure}
\begin{table}[h]
    \centering
    \begin{tabular}{|p{3cm}|p{3cm}|p{3cm}|p{3cm}|p{3cm}|}
        \toprule
        \hline
        \textbf{Factor} & \textbf{Acetone} & \textbf{Stain Remover} & \textbf{Isopropyl Alcohol} & \textbf{Liquid Washing Soap} \\
        \midrule
        \hline
        Suitability for oil-based stains & Excellent & Good & Good & Moderate \\
        Amount of solvent & 5-10 mL, test in inconspicuous area first & Follow product instructions (5-15 mL typical) & 5-10 mL, test in inconspicuous area first & Apply directly to stain or dilute 5-10 mL in water for pre-treatment \\
        \hline
        Cleaning time & 5-10 minutes & 10-15 minutes & 5-10 minutes & Depends on washing cycle time \\
        \hline
        Scrubbing intensity & Light scrubbing with soft cloth & Light to moderate scrubbing & Light scrubbing with soft cloth & Moderate scrubbing with brush or hands \\
        \hline
        Evaporation/drying time & Evaporates quickly (5-10 minutes) & Dries moderately fast (15-30 minutes) & Evaporates quickly (5-10 minutes) & Depends on fabric and drying method \\
        \hline
        Safety & Flammable, use with caution and good ventilation & May contain harsh chemicals, follow product instructions & Flammable, use with caution and good ventilation & Generally safe, but test on inconspicuous area first \\
        \hline
        Fabric suitability & Works well on most fabrics, but test first on delicate fabrics & Check product label for fabric compatibility & Works well on most fabrics, but test first on delicate fabrics & Suitable for washable fabrics \\
        \hline
        \bottomrule
    \end{tabular}
    \caption{Comparison of Different Solvents for Stain Removal}
    \label{tab:solvent_comparison}
\end{table}
\clearpage
\section{Specifications}\label{sec:specs}
\subsection{Energy Specifications}\label{sec:energyspecs}

\begin{center}
\begin{tabular}{|p{8cm}|p{3cm}|}
    % \hline
    \multicolumn{2}{}{\textbf{Roller}} \\
    \hline
    Maximum Operating Voltage & 36 V \\
    \hline
    Maximum Current rating (per channel) & 600 mA \\
    \hline
\end{tabular}
\end{center}
\begin{center}
\begin{tabular}{|p{8cm}|p{3cm}|}
    % \hline
    \multicolumn{2}{}{\textbf{Sprinkler}} \\
    \hline
    \gls{Adapter Voltage Rating} & 12-15 V \\
    \hline
    Maximum Current Rating & 1.2 A \\
    \hline
    Continuous Current Rating & 700 mA \\
    \hline
\end{tabular}
\end{center}
\begin{center}
\begin{tabular}{|p{8cm}|p{3cm}|}
    % \hline
    \multicolumn{2}{}{\textbf{Blower}} \\
    \hline
    DC Motor Voltage & 9,12,15 V \\
    \hline
    Potentiometer Resistance & 200 Ohm \\
    \hline
    Potentiometer Limiting Element Voltage & 250 V \\
    \hline
    Battery Power Supply & 1700 mOhm \\
    \hline
    Battery Frequency & 1kHz \\
    \hline
    Battery Impedance & 9 V \\
    \hline
\end{tabular}
\end{center}
\begin{center}
\begin{tabular}{|p{8cm}|p{3cm}|}
    % \hline
    \multicolumn{2}{}{\textbf{Dryer}} \\
    \hline
    Voltage  & 220V \\
    \hline
\end{tabular}
\end{center}


\subsection{Space Specifications}\label{sec:spacespecs}
\begin{center}
\begin{tabular}{|p{8cm}|p{3cm}|}
    % \hline
    \multicolumn{2}{}{\textbf{Base}} \\
    \hline
    Length & 100 units \\
    \hline
    Breadth & 30 units \\
    \hline
\end{tabular}
\end{center}
\begin{center}
\begin{tabular}{|p{8cm}|p{3cm}|}
    % \hline
    \multicolumn{2}{}{\textbf{Roller}} \\
    \hline
    Diameter & 8 units \\
    \hline
    Length & 25 units \\
    \hline
\end{tabular}
\end{center}
\begin{center}
\begin{tabular}{|p{8cm}|p{3cm}|}
    % \hline
    \multicolumn{2}{}{\textbf{N-Box}} \\
    \hline
    Length & 70 units \\
    \hline
    Breadth & 25 units \\
    \hline
\end{tabular}
\end{center}

\subsection{Power Specifications}\label{sec:powspecs}
\begin{center}
\begin{tabular}{|p{8cm}|p{3cm}|}
    % \hline
    \multicolumn{2}{}{\textbf{Roller}} \\
    \hline
    Power rating of Prototype \index{\gls{Prototype}} & 21.6 W \\
    \hline
\end{tabular}
\end{center}
\begin{center}
\begin{tabular}{|p{8cm}|p{3cm}|}
    % \hline
    \multicolumn{2}{}{\textbf{Sprinkler}} \\
    \hline
    Wattage & 18W \\
    \hline
\end{tabular}
\end{center}
\begin{center}
\begin{tabular}{|p{8cm}|p{3cm}|}
    % \hline
    \multicolumn{2}{}{\textbf{Blower}} \\
    \hline
    DC Motor Power  & 300W \\
    \hline
    DC Motor Speed  & 7000RPM \\
    \hline
    BC547 Max collector Dissipation   & 1.5W \\
    \hline
    Resistor (100M ohms) Power Rating  & 0.25-0.5W \\
    \hline
\end{tabular}
\end{center}
\begin{center}
\begin{tabular}{|p{8cm}|p{3cm}|}
    % \hline
    \multicolumn{2}{}{\textbf{Dryer}} \\
    \hline
    Actual machine requirement & 4-6 KW \\
    \hline
    For Prototype & 1000-2000W \\
    \hline
\end{tabular}
\end{center}

\subsection{Cost Specifications}\label{sec:costspecs}
\begin{center}
\begin{tabular}{|p{8cm}|p{3cm}|}
    % \hline
    \multicolumn{2}{}{\textbf{Roller}} \\
    \hline
    Prototype Average Cost & 200 Rs \\
    \hline
\end{tabular}
\end{center}
\begin{center}
\begin{tabular}{|p{8cm}|p{3cm}|}
    % \hline
    \multicolumn{2}{}{\textbf{Sprinkler}} \\
    \hline
    Arduino & 2500 Rs \\
    \hline
    \gls{Nozzle} & 500 Rs \\
    \hline
    Resistors & 10-20 Rs \\
    \hline
    Breadboard & 90 Rs \\
    \hline
    Sensors & 200 Rs \\
    \hline
    Connecting Wires & 150 Rs \\
    \hline
    NMOS & 70 Rs \\
    \hline
    Solvent Pump & 200 Rs \\
    \hline
\end{tabular}
\end{center}
\begin{center}
\begin{tabular}{|p{8cm}|p{3cm}|}
    % \hline
    \multicolumn{2}{}{\textbf{Scrubber}} \\
    \hline
    Cost per meter & 300-400 Rs \\
    \hline
    No of scrubbers & 200 Ohm \\
    \hline
\end{tabular}
\end{center}
\begin{center}
\begin{tabular}{|p{8cm}|p{3cm}|}
    % \hline
    \multicolumn{2}{}{\textbf{Blower}} \\
    \hline
    Arduino & 2550 Rs \\
    \hline
    Resistor(100 Ohms) & 1 Rs \\
    \hline
    Transistor & 1 Rs \\
    \hline
    Potentiometer & 17 Rs \\
    \hline
    9V battery & 300 Rs \\
    \hline
    DC Motor & 260 Rs \\
    \hline
\end{tabular}
\end{center}
\begin{center}
\begin{tabular}{|p{8cm}|p{3cm}|}
    % \hline
    \multicolumn{2}{}{\textbf{Dryer}} \\
    \hline
    Actual Machine Cost & 100,000 Rs \\
    \hline
    Prototype cost & 800 Rs \\
    \hline
\end{tabular}
\end{center}
\subsection{Performance Specifications}\label{sec:perfspecs}
\subsubsection*{Roller}
        \begin{itemize}[label=$\bullet$]
        \item Rolling Capacity: $\sim$60-80 meters/min.
    \item Rewind Diameter: 1.8-2 meters.
    \item The driver module for the prototype can be used simultaneously by other motors in the system.
    \item The cost of the actual machine can be reduced significantly because it would be integrated with other components.
    \item If a small DC motor is used, a no-load speed of $\sim$9000 rpm, and a loaded speed of $\sim$5000 rpm can be achieved in the prototype. 
        \end{itemize}
    \subsubsection*{Soap Water Sprinkler}
        \begin{itemize}[label=$\bullet$]
        \item HCSR04 Ultrasonic Sensor will check if there is any object placed before the dispenser. A solenoid valve will be used to control the flow of water by energising and deenergising. 
        \item Arduino controls the operation of the water pump. It also controls the flow rate and directions of water.
        \end{itemize}
    \subsubsection*{Scrubber}
        \begin{itemize}[label=$\bullet$]
        \item Stationary scrubbers/brushes remain fixed during operation, providing a stable cleaning surface. 
        \item The static design allows for controlled and targeted cleaning of fabrics, ensuring uniform and effective removal of dirt and contaminants. 
        \end{itemize}
    \subsubsection*{Blower}
        \begin{itemize}[label=$\bullet$]
        \item Control Mechanism \index{\gls{Control Mechanism}}: Potentiometer for DC motor speed control 
        \item Efficiency: Improved motor control using a transistor to prevent overloading and overheating 
        \item Lifespan: Enhanced motor lifespan due to efficient control using the transistor 
        \item Compatibility \index{\gls{Compatibility}}: Multiple DC motors can be connected with slight wiring modifications 
        \end{itemize}
    \subsubsection*{Dryer}
        \begin{itemize}[label=$\bullet$]
        \item Control Mechanism: Arduino, a temperature sensor (thermocouple), a relay module, battery and bunch of connecting wires will be used 
        \item Efficiency \index{\gls{Efficiency}}: Most efficient counter current configuration will be used 
        \item Power requirements will be optimized on the basis of mathematical modelling and simulations done to observe the humidity content with rate of air flow and power input to heater and blower 
        \end{itemize}
\subsection{Manpower Specifications}\label{sec:mpspecs}
\begin{center}
\begin{tabular}{|p{8cm}|p{3cm}|}
    % \hline
    \multicolumn{2}{}{\textbf{}} \\
    \hline
    \textbf{Team} & \textbf{Man Hours} \\
    \hline
    Research Team & 82 Hours \\
    \hline
    Electrical Team & 60 Hours \\
    \hline
    Fabrication Team & 13.5 Hours \\
    \hline
    Documentation Team & 127 Hours \\
    \hline
    Consultant & 2 Hours \\
    \hline
\end{tabular}
\end{center}
\begin{figure}[h]
    \centering
    \includegraphics[width=1.1\textwidth]{logos/workd.png}
    \caption{work distribution mind map}
    \label{fig:specsmindmap}
\end{figure}

\subsection{Milestone Specifications}\label{sec:msspecs}

\subsubsection*{Phase 1 (25\%)}
\begin{enumerate}
    \item[1.] Outline and ideation of the proposed design.
    \item[2.] Learning the skills required for the project.
    \item[3.] Formation and organization of a team for the project.
    \item[4.] Empirical research on the idea and feasibility of the project.
    \item[5.] Visualization of the initial theorized model through basic diagrams on AutoCAD.
\end{enumerate}
\textbf{TRL 2:} Outlining the proposed design, learning skills, and formation of a team indicates the initiation of the project. The use of AutoCAD for basic visualization enhances the maturity by translating ideas into tangible forms, although it's still in the early stages of development involving ideation and conceptualization.

\subsubsection*{Phase 2 (50\%)}
\begin{enumerate}
    \item[1.] Realization of the components required in the design.
    \item[2.] Trial and error checks on the code for the mechanisms.
    \item[3.] Purchasing the required chemicals and inventory to realize the mechanisms.
    \item[4.] Testing the chemicals used on the cloth.
    \item[5.] Finalized design of the visualized model using FreeCAD.
\end{enumerate}
\textbf{TRL 3:} Realizing components, conducting trial and error checks on the code, purchasing required materials, and finalizing the design using ‘FreeCAD’ signify a transition from the early stages to a more mature state. The culmination of these activities indicates a significant advancement in technology readiness, approaching the stage where it can be practically implemented.

\subsubsection*{Phase 3 (75\%)}
\begin{enumerate}
    \item[1.] Showing some individual components, which are completely ready and are in working condition, to the client.
    \item[2.] Finalizing component circuits and codes.
    \item[3.] Including details (dimensions and location) of all the small parts in the model.
\end{enumerate}
\textbf{TRL 4:} Demonstrating fully functional components to the client, finalizing circuits and codes, and providing detailed specifications represent a high level of maturity. This milestone is marked by a readiness for deployment, with working components and comprehensive documentation that can lead to the assembly of a full prototype.

\subsubsection*{Phase 4 (100\%)}
\begin{enumerate}
    \item[1.] Assembling all the different components together to make a full working prototype.
    \item[2.] Live demonstration to the client by using a sample cloth with edges stained with oil and grease, to mimic a just-manufactured cotton cloth.
    \item[3.] Assessment of the model by the quality of cleaning and the time taken.
\end{enumerate}
\textbf{TRL 5:} Assembling all components into a working prototype, conducting a live demonstration using a sample cloth, and assessing performance in real-world conditions with actual stains indicate a high level of readiness for practical application and deployment. The technology has progressed to the point where it can be reliably demonstrated and evaluated in real-world scenarios, signaling a mature state.
% \section{Design}\label{sec:design}
% \section{Closure}\label{sec:closure}
% \section{Reuse}\label{sec:reuse}
% \end{section{Reuse}}

\newpage
\thispagestyle{fancy} % Apply the "fancy" page style only to this page
\rhead{References}
\section*{References}\phantomsection\label{sec:references}
    % \begin{itemize}
    %   \item {\href{https://www.matec-conferences.org/10.1051/matecconf/201822004007}{https://www.matec-conferences.org/10.1051/matecconf/201822004007}}\cite{aqualogic_hot_2023}
    %   \item {\href{https://aqualogic.com.au/blog/hot-warm-cold-wash-make-right-choice-laundry/}{https://aqualogic.com.au/blog/hot-warm-cold-wash-make-right-choice-laundry/}}\cite{chailoet_analytical_2018}
    %   \item {\href{https://www.iqsdirectory.com/articles/dryer/types-of-dryers.html}{https://www.iqsdirectory.com/articles/dryer/types-of-dryers.html}}\cite{cleaners_dry_2020}
    %   \item {\href{https://www.testextextile.com/textile-pretreatment-processes-singeing-desizing-scouring-bleaching-mercerizing/}{https://www.testextextile.com/textile-pretreatment-processes-singeing-desizing-scouring-bleaching-mercerizing/}}
    %   \item {\href{https://www.cda.co.uk/laundry/spin-explained/,%20https://www.cda.co.uk/laundry/spin-explained/}{https://www.cda.co.uk/laundry/spin-explained/,\%20https://www.cda.co.uk/laundry/spin-explained/}}
    %   \item {\href{https://www.textileadvisor.com/2019/04/oil-marks-stain-marks-dust-marks-and.html}{https://www.textileadvisor.com/2019/04/oil-marks-stain-marks-dust-marks-and.html}}
    %   \item {\href{https://www.gentlemansgazette.com/how-to-remove-stains-from-any-garment/}{https://www.gentlemansgazette.com/how-to-remove-stains-from-any-garment/}}
    % \end{itemize}
\begin{itemize}
    \item \cite{aqualogic_hot_2023} - \fullcite{aqualogic_hot_2023}
    \item \cite{chailoet_analytical_2018} - \fullcite{chailoet_analytical_2018}
    \item \cite{cleaners_dry_2020} - \fullcite{cleaners_dry_2020}
    \item \cite{gentlemans_gazette_how_2019} - \fullcite{gentlemans_gazette_how_2019}
    \item \cite{green_agitated_nodate} - \fullcite{green_agitated_nodate}
    \item \cite{kiron_classification_2014} - \fullcite{kiron_classification_2014}
    \item \cite{noauthor__nodate} - \fullcite{noauthor__nodate}
    \item \cite{noauthor_16_nodate} - \fullcite{noauthor_16_nodate}
    \item \cite{noauthor_54_nodate} - \fullcite{noauthor_54_nodate}
    \item \cite{noauthor_arduino_nodate} - \fullcite{noauthor_arduino_nodate}
    \item \cite{noauthor_automatic_nodate} - \fullcite{noauthor_automatic_nodate}
    \item \cite{noauthor_buy_nodate} - \fullcite{noauthor_buy_nodate}
    \item \cite{noauthor_buy_nodate-1} - \fullcite{noauthor_buy_nodate-1}
    \item \cite{noauthor_buy_nodate-2} - \fullcite{noauthor_buy_nodate-2}
    \item \cite{noauthor_buy_nodate-3} - \fullcite{noauthor_buy_nodate-3}
    \item \cite{noauthor_circuit_nodate-3} - \fullcite{noauthor_circuit_nodate-3}
    \item \cite{noauthor_customized_nodate} - \fullcite{noauthor_customized_nodate}
    \item \cite{noauthor_does_nodate} - \fullcite{noauthor_does_nodate}
    \item \cite{noauthor_what_nodate} - \fullcite{noauthor_what_nodate}
    \item \cite{noauthor_how_2019} - \fullcite{noauthor_how_2019}
    \item \cite{noauthor_garment_nodate} - \fullcite{noauthor_garment_nodate}
    % \item \cite{noauthor_tunnel_nodate} - \fullcite{noauthor_tunnel_nodate}
    % \item \cite{noauthor_zyme_nodate} - \fullcite{noauthor_zyme_nodate}
    % \item \cite{noauthor_zyme_nodate} - \fullcite{noauthor_zyme_nodate}
    
    
    % \item \cite{chailoet_analytical_2018} - \fullcite{chailoet_analytical_2018}
    % \item \cite{cleaners_dry_2020} - \fullcite{cleaners_dry_2020}
    % \item \cite{gentlemans_gazette_how_2019} - \fullcite{gentlemans_gazette_how_2019}
    % \item \cite{green_agitated_nodate} - \fullcite{green_agitated_nodate}
    % \item \cite{green_agitated_nodate} - \fullcite{green_agitated_nodate}
    % \item \cite{green_agitated_nodate} - \fullcite{green_agitated_nodate}
    % \item \cite{green_agitated_nodate} - \fullcite{green_agitated_nodate}
    % \item \cite{green_agitated_nodate} - \fullcite{green_agitated_nodate}
    % \item \cite{green_agitated_nodate} - \fullcite{green_agitated_nodate}
    % \item \cite{green_agitated_nodate} - \fullcite{green_agitated_nodate}
    % \item \cite{green_agitated_nodate} - \fullcite{green_agitated_nodate}
    % \item \cite{green_agitated_nodate} - \fullcite{green_agitated_nodate}
    % \item \cite{green_agitated_nodate} - \fullcite{green_agitated_nodate}
    % \item \cite{green_agitated_nodate} - \fullcite{green_agitated_nodate}
    % \item \cite{green_agitated_nodate} - \fullcite{green_agitated_nodate}
    % \item \cite{green_agitated_nodate} - \fullcite{green_agitated_nodate}
    % Add more citations as needed
\end{itemize}
    
\clearpage
% \section*{}
\phantomsection
\printindex\label{sec:myindex}
% \clearpage
% \newpage
% \section*{Glossary}\label{sec: glossary}
\thispagestyle{fancy} % Apply the "fancy" page style only to this page
\rhead{Glossary}
\renewcommand{\glossaryname}{Glossary}
\printglossaries\label{sec: glossary}
% \section{Glossary}

% Use the \gls{} command to reference glossary entries
% \gls{Acrylonitrile butadiene styrene} is a common thermoplastic. \gls{Affix} refers to securely attaching or fastening.

% A \gls{Air Blower} is a device that produces a stream of air under pressure. In manufacturing, products are often assembled on an \gls{Assembly Line}.

% The \gls{Automated Readability Index} is a formula used to calculate the understandability of a text.

% \gls{Baking soda} is an alkaline powder used for odor absorption and mild stain removal. \gls{Bleaching Powder} is a chemical
% \printglossaries
% \newpage
\section*{Appendix}\phantomsection\label{sec:appendix}
\begin{table}[h]
      \centering
      \begin{tabular}{|c|c|}
        \hline
        \textbf{Document type} & Private Release \\
        \hline
        \textbf{Document Authorised by} & Ayush Sharma \\
        \hline
        \textbf{Publication Date} & 21/01/2024 \\
        \hline
        \textbf{Version Number} & v1.5.1 \\
        \hline
        \textbf{GitHub Repo Details} & \href{https://github.com/naunidhsingh03/ELP305-Tribe-D}{ https://github.com/ELP305-Tribe-D} \\
        \hline
      \end{tabular}
      \caption{Document ID}
      \label{tab:documentid}
    \end{table}
\begin{table}
      \centering
      \begin{tabular}{|c|c|}
        \hline
        \textbf{Word Count}: & 3453 \\
        \hline
        \textbf{Number of Sentence} & 180 \\
        \hline
        \textbf{Number of Characters(Without Spaces)} & 19288 \\
        \hline
      \end{tabular}
      \caption{Document Statistics}
      \label{tab:documentstats}
    \end{table}

    \begin{table}
      \centering
      \begin{tabular}{|c|c|}
        \hline
        \textbf{Readability (0-100)} & 62.5 \\
        \hline
        \textbf{Gunning Fog Index (0-20)} & 13.9 \\
        \hline
        \textbf{Flesch Reading Ease (0-100)} & 52 \\
        \hline
        \textbf{Coleman Liau Index (0-17+)} & 14.33 \\
        \hline
      \end{tabular}
      \caption{Readability Indices}
      \label{tab:readability}
    \end{table}
\newpage

\textbf{}\newline
\textbf{Readibility}\newline
\textit{Score Range: 0-100}\newline
Explanation: The Readability (0-100) score, specifically the Flesch Reading Ease score, measures how easy or difficult a piece of text is to read. The higher the score, the easier the text is to understand. The score is calculated based on the average sentence length and the average number of syllables per word in the text.
\textbf{}\newline
\textbf{Gunning Fog Readability}\newline
\textit{Score Range: 0-20}\newline
Explanation: The Gunning Fog Index estimates the years of formal education needed to understand a piece of text. The higher the index, the more difficult the text is to comprehend. It considers the average sentence length and the percentage of complex words (words with three or more syllables).
\textbf{}\newline
\textbf{Flesch Reading Ease}\newline
\textit{Score Range: 0-100}\newline
Explanation: The Flesch Reading Ease score is a measure of how easy or difficult a piece of text is to read. The higher the score, the easier the text is to understand. The formula takes into account the average sentence length and the average number of syllables per word.
\textbf{}\newline
\textbf{Coleman Liau Readability Index}\newline
\textit{Score Range: 0-17+}\newline
Explanation: The Coleman Liau Index determines the readability of a text by using characters per word and words per sentence. It provides an estimate of the U.S. school grade level required to comprehend the text. Higher scores indicate more difficult readability.


\rule{\linewidth}{0.5pt}
\end{document}

  